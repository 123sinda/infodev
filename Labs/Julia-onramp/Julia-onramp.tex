% !TEX TS-program = xelatex
\documentclass[11pt]{article}

    \usepackage[breakable]{tcolorbox}
    \usepackage{parskip} % Stop auto-indenting (to mimic markdown behaviour)
    \usepackage{libertine}

    % Basic figure setup, for now with no caption control since it's done
    % automatically by Pandoc (which extracts ![](path) syntax from Markdown).
    \usepackage{graphicx}
    % Maintain compatibility with old templates. Remove in nbconvert 6.0
    \let\Oldincludegraphics\includegraphics
    % Ensure that by default, figures have no caption (until we provide a
    % proper Figure object with a Caption API and a way to capture that
    % in the conversion process - todo).
    \usepackage{caption}
    \DeclareCaptionFormat{nocaption}{}
    \captionsetup{format=nocaption,aboveskip=0pt,belowskip=0pt}

    \usepackage{float}
    \floatplacement{figure}{H} % forces figures to be placed at the correct location
    \usepackage{xcolor} % Allow colors to be defined
    \usepackage{enumerate} % Needed for markdown enumerations to work
    \usepackage{geometry} % Used to adjust the document margins
    \usepackage{amsmath} % Equations
    \usepackage{amssymb} % Equations
    \usepackage{textcomp} % defines textquotesingle
    % Hack from http://tex.stackexchange.com/a/47451/13684:
    \AtBeginDocument{%
        \def\PYZsq{\textquotesingle}% Upright quotes in Pygmentized code
    }
    \usepackage{upquote} % Upright quotes for verbatim code
    \usepackage{eurosym} % defines \euro

    \usepackage{iftex}
    \ifPDFTeX
        \usepackage[T1]{fontenc}
        \IfFileExists{alphabeta.sty}{
              \usepackage{alphabeta}
          }{
              \usepackage[mathletters]{ucs}
              \usepackage[utf8x]{inputenc}
          }
    \else
        \usepackage{fontspec}
        \usepackage{unicode-math}
    \fi

    \usepackage{fancyvrb} % verbatim replacement that allows latex
    \usepackage{grffile} % extends the file name processing of package graphics
                         % to support a larger range
    \makeatletter % fix for old versions of grffile with XeLaTeX
    \@ifpackagelater{grffile}{2019/11/01}
    {
      % Do nothing on new versions
    }
    {
      \def\Gread@@xetex#1{%
        \IfFileExists{"\Gin@base".bb}%
        {\Gread@eps{\Gin@base.bb}}%
        {\Gread@@xetex@aux#1}%
      }
    }
    \makeatother
    \usepackage[Export]{adjustbox} % Used to constrain images to a maximum size
    \adjustboxset{max size={0.9\linewidth}{0.9\paperheight}}

    % The hyperref package gives us a pdf with properly built
    % internal navigation ('pdf bookmarks' for the table of contents,
    % internal cross-reference links, web links for URLs, etc.)
    \usepackage{hyperref}
    % The default LaTeX title has an obnoxious amount of whitespace. By default,
    % titling removes some of it. It also provides customization options.
    \usepackage{titling}
    \usepackage{longtable} % longtable support required by pandoc >1.10
    \usepackage{booktabs}  % table support for pandoc > 1.12.2
    \usepackage{array}     % table support for pandoc >= 2.11.3
    \usepackage{calc}      % table minipage width calculation for pandoc >= 2.11.1
    \usepackage[inline]{enumitem} % IRkernel/repr support (it uses the enumerate* environment)
    \usepackage[normalem]{ulem} % ulem is needed to support strikethroughs (\sout)
                                % normalem makes italics be italics, not underlines
    \usepackage{mathrsfs}
    

    
    % Colors for the hyperref package
    \definecolor{urlcolor}{rgb}{0,.145,.698}
    \definecolor{linkcolor}{rgb}{.71,0.21,0.01}
    \definecolor{citecolor}{rgb}{.12,.54,.11}

    % ANSI colors
    \definecolor{ansi-black}{HTML}{3E424D}
    \definecolor{ansi-black-intense}{HTML}{282C36}
    \definecolor{ansi-red}{HTML}{E75C58}
    \definecolor{ansi-red-intense}{HTML}{B22B31}
    \definecolor{ansi-green}{HTML}{00A250}
    \definecolor{ansi-green-intense}{HTML}{007427}
    \definecolor{ansi-yellow}{HTML}{DDB62B}
    \definecolor{ansi-yellow-intense}{HTML}{B27D12}
    \definecolor{ansi-blue}{HTML}{208FFB}
    \definecolor{ansi-blue-intense}{HTML}{0065CA}
    \definecolor{ansi-magenta}{HTML}{D160C4}
    \definecolor{ansi-magenta-intense}{HTML}{A03196}
    \definecolor{ansi-cyan}{HTML}{60C6C8}
    \definecolor{ansi-cyan-intense}{HTML}{258F8F}
    \definecolor{ansi-white}{HTML}{C5C1B4}
    \definecolor{ansi-white-intense}{HTML}{A1A6B2}
    \definecolor{ansi-default-inverse-fg}{HTML}{FFFFFF}
    \definecolor{ansi-default-inverse-bg}{HTML}{000000}

    % common color for the border for error outputs.
    \definecolor{outerrorbackground}{HTML}{FFDFDF}

    % commands and environments needed by pandoc snippets
    % extracted from the output of `pandoc -s`
    \providecommand{\tightlist}{%
      \setlength{\itemsep}{0pt}\setlength{\parskip}{0pt}}
    \DefineVerbatimEnvironment{Highlighting}{Verbatim}{commandchars=\\\{\}}
    % Add ',fontsize=\small' for more characters per line
    \newenvironment{Shaded}{}{}
    \newcommand{\KeywordTok}[1]{\textcolor[rgb]{0.00,0.44,0.13}{\textbf{{#1}}}}
    \newcommand{\DataTypeTok}[1]{\textcolor[rgb]{0.56,0.13,0.00}{{#1}}}
    \newcommand{\DecValTok}[1]{\textcolor[rgb]{0.25,0.63,0.44}{{#1}}}
    \newcommand{\BaseNTok}[1]{\textcolor[rgb]{0.25,0.63,0.44}{{#1}}}
    \newcommand{\FloatTok}[1]{\textcolor[rgb]{0.25,0.63,0.44}{{#1}}}
    \newcommand{\CharTok}[1]{\textcolor[rgb]{0.25,0.44,0.63}{{#1}}}
    \newcommand{\StringTok}[1]{\textcolor[rgb]{0.25,0.44,0.63}{{#1}}}
    \newcommand{\CommentTok}[1]{\textcolor[rgb]{0.38,0.63,0.69}{\textit{{#1}}}}
    \newcommand{\OtherTok}[1]{\textcolor[rgb]{0.00,0.44,0.13}{{#1}}}
    \newcommand{\AlertTok}[1]{\textcolor[rgb]{1.00,0.00,0.00}{\textbf{{#1}}}}
    \newcommand{\FunctionTok}[1]{\textcolor[rgb]{0.02,0.16,0.49}{{#1}}}
    \newcommand{\RegionMarkerTok}[1]{{#1}}
    \newcommand{\ErrorTok}[1]{\textcolor[rgb]{1.00,0.00,0.00}{\textbf{{#1}}}}
    \newcommand{\NormalTok}[1]{{#1}}

    % Additional commands for more recent versions of Pandoc
    \newcommand{\ConstantTok}[1]{\textcolor[rgb]{0.53,0.00,0.00}{{#1}}}
    \newcommand{\SpecialCharTok}[1]{\textcolor[rgb]{0.25,0.44,0.63}{{#1}}}
    \newcommand{\VerbatimStringTok}[1]{\textcolor[rgb]{0.25,0.44,0.63}{{#1}}}
    \newcommand{\SpecialStringTok}[1]{\textcolor[rgb]{0.73,0.40,0.53}{{#1}}}
    \newcommand{\ImportTok}[1]{{#1}}
    \newcommand{\DocumentationTok}[1]{\textcolor[rgb]{0.73,0.13,0.13}{\textit{{#1}}}}
    \newcommand{\AnnotationTok}[1]{\textcolor[rgb]{0.38,0.63,0.69}{\textbf{\textit{{#1}}}}}
    \newcommand{\CommentVarTok}[1]{\textcolor[rgb]{0.38,0.63,0.69}{\textbf{\textit{{#1}}}}}
    \newcommand{\VariableTok}[1]{\textcolor[rgb]{0.10,0.09,0.49}{{#1}}}
    \newcommand{\ControlFlowTok}[1]{\textcolor[rgb]{0.00,0.44,0.13}{\textbf{{#1}}}}
    \newcommand{\OperatorTok}[1]{\textcolor[rgb]{0.40,0.40,0.40}{{#1}}}
    \newcommand{\BuiltInTok}[1]{{#1}}
    \newcommand{\ExtensionTok}[1]{{#1}}
    \newcommand{\PreprocessorTok}[1]{\textcolor[rgb]{0.74,0.48,0.00}{{#1}}}
    \newcommand{\AttributeTok}[1]{\textcolor[rgb]{0.49,0.56,0.16}{{#1}}}
    \newcommand{\InformationTok}[1]{\textcolor[rgb]{0.38,0.63,0.69}{\textbf{\textit{{#1}}}}}
    \newcommand{\WarningTok}[1]{\textcolor[rgb]{0.38,0.63,0.69}{\textbf{\textit{{#1}}}}}


    % Define a nice break command that doesn't care if a line doesn't already
    % exist.
    \def\br{\hspace*{\fill} \\* }
    % Math Jax compatibility definitions
    \def\gt{>}
    \def\lt{<}
    \let\Oldtex\TeX
    \let\Oldlatex\LaTeX
    \renewcommand{\TeX}{\textrm{\Oldtex}}
    \renewcommand{\LaTeX}{\textrm{\Oldlatex}}
    % Document parameters
    % Document title
    \title{Julia Onramp}
    
    \author{A. Mhamdi}
    
    
    
% Pygments definitions
\makeatletter
\def\PY@reset{\let\PY@it=\relax \let\PY@bf=\relax%
    \let\PY@ul=\relax \let\PY@tc=\relax%
    \let\PY@bc=\relax \let\PY@ff=\relax}
\def\PY@tok#1{\csname PY@tok@#1\endcsname}
\def\PY@toks#1+{\ifx\relax#1\empty\else%
    \PY@tok{#1}\expandafter\PY@toks\fi}
\def\PY@do#1{\PY@bc{\PY@tc{\PY@ul{%
    \PY@it{\PY@bf{\PY@ff{#1}}}}}}}
\def\PY#1#2{\PY@reset\PY@toks#1+\relax+\PY@do{#2}}

\@namedef{PY@tok@w}{\def\PY@tc##1{\textcolor[rgb]{0.73,0.73,0.73}{##1}}}
\@namedef{PY@tok@c}{\let\PY@it=\textit\def\PY@tc##1{\textcolor[rgb]{0.24,0.48,0.48}{##1}}}
\@namedef{PY@tok@cp}{\def\PY@tc##1{\textcolor[rgb]{0.61,0.40,0.00}{##1}}}
\@namedef{PY@tok@k}{\let\PY@bf=\textbf\def\PY@tc##1{\textcolor[rgb]{0.00,0.50,0.00}{##1}}}
\@namedef{PY@tok@kp}{\def\PY@tc##1{\textcolor[rgb]{0.00,0.50,0.00}{##1}}}
\@namedef{PY@tok@kt}{\def\PY@tc##1{\textcolor[rgb]{0.69,0.00,0.25}{##1}}}
\@namedef{PY@tok@o}{\def\PY@tc##1{\textcolor[rgb]{0.40,0.40,0.40}{##1}}}
\@namedef{PY@tok@ow}{\let\PY@bf=\textbf\def\PY@tc##1{\textcolor[rgb]{0.67,0.13,1.00}{##1}}}
\@namedef{PY@tok@nb}{\def\PY@tc##1{\textcolor[rgb]{0.00,0.50,0.00}{##1}}}
\@namedef{PY@tok@nf}{\def\PY@tc##1{\textcolor[rgb]{0.00,0.00,1.00}{##1}}}
\@namedef{PY@tok@nc}{\let\PY@bf=\textbf\def\PY@tc##1{\textcolor[rgb]{0.00,0.00,1.00}{##1}}}
\@namedef{PY@tok@nn}{\let\PY@bf=\textbf\def\PY@tc##1{\textcolor[rgb]{0.00,0.00,1.00}{##1}}}
\@namedef{PY@tok@ne}{\let\PY@bf=\textbf\def\PY@tc##1{\textcolor[rgb]{0.80,0.25,0.22}{##1}}}
\@namedef{PY@tok@nv}{\def\PY@tc##1{\textcolor[rgb]{0.10,0.09,0.49}{##1}}}
\@namedef{PY@tok@no}{\def\PY@tc##1{\textcolor[rgb]{0.53,0.00,0.00}{##1}}}
\@namedef{PY@tok@nl}{\def\PY@tc##1{\textcolor[rgb]{0.46,0.46,0.00}{##1}}}
\@namedef{PY@tok@ni}{\let\PY@bf=\textbf\def\PY@tc##1{\textcolor[rgb]{0.44,0.44,0.44}{##1}}}
\@namedef{PY@tok@na}{\def\PY@tc##1{\textcolor[rgb]{0.41,0.47,0.13}{##1}}}
\@namedef{PY@tok@nt}{\let\PY@bf=\textbf\def\PY@tc##1{\textcolor[rgb]{0.00,0.50,0.00}{##1}}}
\@namedef{PY@tok@nd}{\def\PY@tc##1{\textcolor[rgb]{0.67,0.13,1.00}{##1}}}
\@namedef{PY@tok@s}{\def\PY@tc##1{\textcolor[rgb]{0.73,0.13,0.13}{##1}}}
\@namedef{PY@tok@sd}{\let\PY@it=\textit\def\PY@tc##1{\textcolor[rgb]{0.73,0.13,0.13}{##1}}}
\@namedef{PY@tok@si}{\let\PY@bf=\textbf\def\PY@tc##1{\textcolor[rgb]{0.64,0.35,0.47}{##1}}}
\@namedef{PY@tok@se}{\let\PY@bf=\textbf\def\PY@tc##1{\textcolor[rgb]{0.67,0.36,0.12}{##1}}}
\@namedef{PY@tok@sr}{\def\PY@tc##1{\textcolor[rgb]{0.64,0.35,0.47}{##1}}}
\@namedef{PY@tok@ss}{\def\PY@tc##1{\textcolor[rgb]{0.10,0.09,0.49}{##1}}}
\@namedef{PY@tok@sx}{\def\PY@tc##1{\textcolor[rgb]{0.00,0.50,0.00}{##1}}}
\@namedef{PY@tok@m}{\def\PY@tc##1{\textcolor[rgb]{0.40,0.40,0.40}{##1}}}
\@namedef{PY@tok@gh}{\let\PY@bf=\textbf\def\PY@tc##1{\textcolor[rgb]{0.00,0.00,0.50}{##1}}}
\@namedef{PY@tok@gu}{\let\PY@bf=\textbf\def\PY@tc##1{\textcolor[rgb]{0.50,0.00,0.50}{##1}}}
\@namedef{PY@tok@gd}{\def\PY@tc##1{\textcolor[rgb]{0.63,0.00,0.00}{##1}}}
\@namedef{PY@tok@gi}{\def\PY@tc##1{\textcolor[rgb]{0.00,0.52,0.00}{##1}}}
\@namedef{PY@tok@gr}{\def\PY@tc##1{\textcolor[rgb]{0.89,0.00,0.00}{##1}}}
\@namedef{PY@tok@ge}{\let\PY@it=\textit}
\@namedef{PY@tok@gs}{\let\PY@bf=\textbf}
\@namedef{PY@tok@gp}{\let\PY@bf=\textbf\def\PY@tc##1{\textcolor[rgb]{0.00,0.00,0.50}{##1}}}
\@namedef{PY@tok@go}{\def\PY@tc##1{\textcolor[rgb]{0.44,0.44,0.44}{##1}}}
\@namedef{PY@tok@gt}{\def\PY@tc##1{\textcolor[rgb]{0.00,0.27,0.87}{##1}}}
\@namedef{PY@tok@err}{\def\PY@bc##1{{\setlength{\fboxsep}{\string -\fboxrule}\fcolorbox[rgb]{1.00,0.00,0.00}{1,1,1}{\strut ##1}}}}
\@namedef{PY@tok@kc}{\let\PY@bf=\textbf\def\PY@tc##1{\textcolor[rgb]{0.00,0.50,0.00}{##1}}}
\@namedef{PY@tok@kd}{\let\PY@bf=\textbf\def\PY@tc##1{\textcolor[rgb]{0.00,0.50,0.00}{##1}}}
\@namedef{PY@tok@kn}{\let\PY@bf=\textbf\def\PY@tc##1{\textcolor[rgb]{0.00,0.50,0.00}{##1}}}
\@namedef{PY@tok@kr}{\let\PY@bf=\textbf\def\PY@tc##1{\textcolor[rgb]{0.00,0.50,0.00}{##1}}}
\@namedef{PY@tok@bp}{\def\PY@tc##1{\textcolor[rgb]{0.00,0.50,0.00}{##1}}}
\@namedef{PY@tok@fm}{\def\PY@tc##1{\textcolor[rgb]{0.00,0.00,1.00}{##1}}}
\@namedef{PY@tok@vc}{\def\PY@tc##1{\textcolor[rgb]{0.10,0.09,0.49}{##1}}}
\@namedef{PY@tok@vg}{\def\PY@tc##1{\textcolor[rgb]{0.10,0.09,0.49}{##1}}}
\@namedef{PY@tok@vi}{\def\PY@tc##1{\textcolor[rgb]{0.10,0.09,0.49}{##1}}}
\@namedef{PY@tok@vm}{\def\PY@tc##1{\textcolor[rgb]{0.10,0.09,0.49}{##1}}}
\@namedef{PY@tok@sa}{\def\PY@tc##1{\textcolor[rgb]{0.73,0.13,0.13}{##1}}}
\@namedef{PY@tok@sb}{\def\PY@tc##1{\textcolor[rgb]{0.73,0.13,0.13}{##1}}}
\@namedef{PY@tok@sc}{\def\PY@tc##1{\textcolor[rgb]{0.73,0.13,0.13}{##1}}}
\@namedef{PY@tok@dl}{\def\PY@tc##1{\textcolor[rgb]{0.73,0.13,0.13}{##1}}}
\@namedef{PY@tok@s2}{\def\PY@tc##1{\textcolor[rgb]{0.73,0.13,0.13}{##1}}}
\@namedef{PY@tok@sh}{\def\PY@tc##1{\textcolor[rgb]{0.73,0.13,0.13}{##1}}}
\@namedef{PY@tok@s1}{\def\PY@tc##1{\textcolor[rgb]{0.73,0.13,0.13}{##1}}}
\@namedef{PY@tok@mb}{\def\PY@tc##1{\textcolor[rgb]{0.40,0.40,0.40}{##1}}}
\@namedef{PY@tok@mf}{\def\PY@tc##1{\textcolor[rgb]{0.40,0.40,0.40}{##1}}}
\@namedef{PY@tok@mh}{\def\PY@tc##1{\textcolor[rgb]{0.40,0.40,0.40}{##1}}}
\@namedef{PY@tok@mi}{\def\PY@tc##1{\textcolor[rgb]{0.40,0.40,0.40}{##1}}}
\@namedef{PY@tok@il}{\def\PY@tc##1{\textcolor[rgb]{0.40,0.40,0.40}{##1}}}
\@namedef{PY@tok@mo}{\def\PY@tc##1{\textcolor[rgb]{0.40,0.40,0.40}{##1}}}
\@namedef{PY@tok@ch}{\let\PY@it=\textit\def\PY@tc##1{\textcolor[rgb]{0.24,0.48,0.48}{##1}}}
\@namedef{PY@tok@cm}{\let\PY@it=\textit\def\PY@tc##1{\textcolor[rgb]{0.24,0.48,0.48}{##1}}}
\@namedef{PY@tok@cpf}{\let\PY@it=\textit\def\PY@tc##1{\textcolor[rgb]{0.24,0.48,0.48}{##1}}}
\@namedef{PY@tok@c1}{\let\PY@it=\textit\def\PY@tc##1{\textcolor[rgb]{0.24,0.48,0.48}{##1}}}
\@namedef{PY@tok@cs}{\let\PY@it=\textit\def\PY@tc##1{\textcolor[rgb]{0.24,0.48,0.48}{##1}}}

\def\PYZbs{\char`\\}
\def\PYZus{\char`\_}
\def\PYZob{\char`\{}
\def\PYZcb{\char`\}}
\def\PYZca{\char`\^}
\def\PYZam{\char`\&}
\def\PYZlt{\char`\<}
\def\PYZgt{\char`\>}
\def\PYZsh{\char`\#}
\def\PYZpc{\char`\%}
\def\PYZdl{\char`\$}
\def\PYZhy{\char`\-}
\def\PYZsq{\char`\'}
\def\PYZdq{\char`\"}
\def\PYZti{\char`\~}
% for compatibility with earlier versions
\def\PYZat{@}
\def\PYZlb{[}
\def\PYZrb{]}
\makeatother


    % For linebreaks inside Verbatim environment from package fancyvrb.
    \makeatletter
        \newbox\Wrappedcontinuationbox
        \newbox\Wrappedvisiblespacebox
        \newcommand*\Wrappedvisiblespace {\textcolor{red}{\textvisiblespace}}
        \newcommand*\Wrappedcontinuationsymbol {\textcolor{red}{\llap{\tiny$\m@th\hookrightarrow$}}}
        \newcommand*\Wrappedcontinuationindent {3ex }
        \newcommand*\Wrappedafterbreak {\kern\Wrappedcontinuationindent\copy\Wrappedcontinuationbox}
        % Take advantage of the already applied Pygments mark-up to insert
        % potential linebreaks for TeX processing.
        %        {, <, #, %, $, ' and ": go to next line.
        %        _, }, ^, &, >, - and ~: stay at end of broken line.
        % Use of \textquotesingle for straight quote.
        \newcommand*\Wrappedbreaksatspecials {%
            \def\PYGZus{\discretionary{\char`\_}{\Wrappedafterbreak}{\char`\_}}%
            \def\PYGZob{\discretionary{}{\Wrappedafterbreak\char`\{}{\char`\{}}%
            \def\PYGZcb{\discretionary{\char`\}}{\Wrappedafterbreak}{\char`\}}}%
            \def\PYGZca{\discretionary{\char`\^}{\Wrappedafterbreak}{\char`\^}}%
            \def\PYGZam{\discretionary{\char`\&}{\Wrappedafterbreak}{\char`\&}}%
            \def\PYGZlt{\discretionary{}{\Wrappedafterbreak\char`\<}{\char`\<}}%
            \def\PYGZgt{\discretionary{\char`\>}{\Wrappedafterbreak}{\char`\>}}%
            \def\PYGZsh{\discretionary{}{\Wrappedafterbreak\char`\#}{\char`\#}}%
            \def\PYGZpc{\discretionary{}{\Wrappedafterbreak\char`\%}{\char`\%}}%
            \def\PYGZdl{\discretionary{}{\Wrappedafterbreak\char`\$}{\char`\$}}%
            \def\PYGZhy{\discretionary{\char`\-}{\Wrappedafterbreak}{\char`\-}}%
            \def\PYGZsq{\discretionary{}{\Wrappedafterbreak\textquotesingle}{\textquotesingle}}%
            \def\PYGZdq{\discretionary{}{\Wrappedafterbreak\char`\"}{\char`\"}}%
            \def\PYGZti{\discretionary{\char`\~}{\Wrappedafterbreak}{\char`\~}}%
        }
        % Some characters . , ; ? ! / are not pygmentized.
        % This macro makes them "active" and they will insert potential linebreaks
        \newcommand*\Wrappedbreaksatpunct {%
            \lccode`\~`\.\lowercase{\def~}{\discretionary{\hbox{\char`\.}}{\Wrappedafterbreak}{\hbox{\char`\.}}}%
            \lccode`\~`\,\lowercase{\def~}{\discretionary{\hbox{\char`\,}}{\Wrappedafterbreak}{\hbox{\char`\,}}}%
            \lccode`\~`\;\lowercase{\def~}{\discretionary{\hbox{\char`\;}}{\Wrappedafterbreak}{\hbox{\char`\;}}}%
            \lccode`\~`\:\lowercase{\def~}{\discretionary{\hbox{\char`\:}}{\Wrappedafterbreak}{\hbox{\char`\:}}}%
            \lccode`\~`\?\lowercase{\def~}{\discretionary{\hbox{\char`\?}}{\Wrappedafterbreak}{\hbox{\char`\?}}}%
            \lccode`\~`\!\lowercase{\def~}{\discretionary{\hbox{\char`\!}}{\Wrappedafterbreak}{\hbox{\char`\!}}}%
            \lccode`\~`\/\lowercase{\def~}{\discretionary{\hbox{\char`\/}}{\Wrappedafterbreak}{\hbox{\char`\/}}}%
            \catcode`\.\active
            \catcode`\,\active
            \catcode`\;\active
            \catcode`\:\active
            \catcode`\?\active
            \catcode`\!\active
            \catcode`\/\active
            \lccode`\~`\~
        }
    \makeatother

    \let\OriginalVerbatim=\Verbatim
    \makeatletter
    \renewcommand{\Verbatim}[1][1]{%
        %\parskip\z@skip
        \sbox\Wrappedcontinuationbox {\Wrappedcontinuationsymbol}%
        \sbox\Wrappedvisiblespacebox {\FV@SetupFont\Wrappedvisiblespace}%
        \def\FancyVerbFormatLine ##1{\hsize\linewidth
            \vtop{\raggedright\hyphenpenalty\z@\exhyphenpenalty\z@
                \doublehyphendemerits\z@\finalhyphendemerits\z@
                \strut ##1\strut}%
        }%
        % If the linebreak is at a space, the latter will be displayed as visible
        % space at end of first line, and a continuation symbol starts next line.
        % Stretch/shrink are however usually zero for typewriter font.
        \def\FV@Space {%
            \nobreak\hskip\z@ plus\fontdimen3\font minus\fontdimen4\font
            \discretionary{\copy\Wrappedvisiblespacebox}{\Wrappedafterbreak}
            {\kern\fontdimen2\font}%
        }%

        % Allow breaks at special characters using \PYG... macros.
        \Wrappedbreaksatspecials
        % Breaks at punctuation characters . , ; ? ! and / need catcode=\active
        \OriginalVerbatim[#1,codes*=\Wrappedbreaksatpunct]%
    }
    \makeatother

    % Exact colors from NB
    \definecolor{incolor}{HTML}{303F9F}
    \definecolor{outcolor}{HTML}{D84315}
    \definecolor{cellborder}{HTML}{CFCFCF}
    \definecolor{cellbackground}{HTML}{F7F7F7}

    % prompt
    \makeatletter
    \newcommand{\boxspacing}{\kern\kvtcb@left@rule\kern\kvtcb@boxsep}
    \makeatother
    \newcommand{\prompt}[4]{
        {\ttfamily\llap{{\color{#2}[#3]:\hspace{3pt}#4}}\vspace{-\baselineskip}}
    }
    

    
    % Prevent overflowing lines due to hard-to-break entities
    \sloppy
    % Setup hyperref package
    \hypersetup{
      breaklinks=true,  % so long urls are correctly broken across lines
      colorlinks=true,
      urlcolor=urlcolor,
      linkcolor=linkcolor,
      citecolor=citecolor,
      }
    % Slightly bigger margins than the latex defaults
    
    \geometry{verbose,tmargin=1in,bmargin=1in,lmargin=1in,rmargin=1in}
    
    

\begin{document}
    
    \maketitle
    
    

  

    \hypertarget{goals}{%
\section{GOALS}\label{goals}}

\begin{itemize}
\tightlist
\item
  Enter commands in \({\tt Julia}\) REPL to create variables and perform
  calculations;
\item
  Write and save programs;
\item
  Use indexing to extract and modify rows, columns, and elements of
  \({\tt Julia}\) tensors.
\end{itemize}

    \({\tt Julia}\) is a standalone program which can be downloaded from
\url{https://julialang.org/downloads/}

    \textbf{Getting around}

    By default, Julia runs in an interactive terminal called the REPL. In
this mode, Some useful commands are: 1. \texttt{\^{}C} aborts execution
1. \texttt{\^{}D} exits Julia 1. \texttt{?} enters help mode 1.
\texttt{;} enters system shell mode 1. \texttt{{]}} enters package
manager mode 1. \texttt{\^{}l} clears screen

    We begin first by activating the environement within the desired folder.

    From the REPL interface, either type

\begin{Shaded}
\begin{Highlighting}[]
\KeywordTok{using}\NormalTok{ Pkg  }
\NormalTok{pkg}\StringTok{"activate ."}
\end{Highlighting}
\end{Shaded}

or access the package mode by typing \texttt{{]}} and simply write

\begin{Shaded}
\begin{Highlighting}[]
\NormalTok{activate .}
\end{Highlighting}
\end{Shaded}

Always within the package mode, to see the full list of installed
packages

\begin{Shaded}
\begin{Highlighting}[]
\NormalTok{st}
\end{Highlighting}
\end{Shaded}

    \begin{tcolorbox}[breakable, size=fbox, boxrule=1pt, pad at break*=1mm,colback=cellbackground, colframe=cellborder]
\prompt{In}{incolor}{1}{\boxspacing}
\begin{Verbatim}[commandchars=\\\{\}]
\PY{k}{using}\PY{+w}{ }\PY{n}{Pkg}
\PY{l+s+sa}{pkg}\PY{l+s}{\PYZdq{}}\PY{l+s}{activate .}\PY{l+s}{\PYZdq{}}
\end{Verbatim}
\end{tcolorbox}

    \begin{Verbatim}[commandchars=\\\{\}]
\textcolor{ansi-green-intense}{\textbf{  Activating}} project at `\textasciitilde{}/MEGA/git-repos/infodev/Codes`
    \end{Verbatim}

    \begin{tcolorbox}[breakable, size=fbox, boxrule=1pt, pad at break*=1mm,colback=cellbackground, colframe=cellborder]
\prompt{In}{incolor}{2}{\boxspacing}
\begin{Verbatim}[commandchars=\\\{\}]
\PY{p}{]}\PY{n}{st}
\end{Verbatim}
\end{tcolorbox}

    \begin{Verbatim}[commandchars=\\\{\}]
\textcolor{ansi-green-intense}{\textbf{Status}} `\textasciitilde{}/MEGA/git-repos/infodev/Codes/Project.toml`
  \textcolor{ansi-black-intense}{[336ed68f] }CSV v0.10.11
  \textcolor{ansi-black-intense}{[a93c6f00] }DataFrames v1.6.1
  \textcolor{ansi-black-intense}{[7073ff75] }IJulia v1.24.2
\textcolor{ansi-green}{⌃} \textcolor{ansi-black-intense}{[ee78f7c6] }Makie v0.19.12
  \textcolor{ansi-black-intense}{[5deeb4b9] }Mousetrap v0.3.1
`https://github.com/clemapfel/mousetrap.jl\#main`
  \textcolor{ansi-black-intense}{[5fb14364] }OhMyREPL v0.5.23
  \textcolor{ansi-black-intense}{[91a5bcdd] }Plots v1.39.0
  \textcolor{ansi-black-intense}{[c3e4b0f8] }Pluto v0.19.32
  \textcolor{ansi-black-intense}{[d6f4376e] }Markdown
\textcolor{ansi-cyan-intense}{\textbf{Info}} Packages marked with \textcolor{ansi-green}{⌃} have new versions
available and may be upgradable.
    \end{Verbatim}

    To add the \textbf{Markdown} package for instance, we write

    \begin{tcolorbox}[breakable, size=fbox, boxrule=1pt, pad at break*=1mm,colback=cellbackground, colframe=cellborder]
\prompt{In}{incolor}{3}{\boxspacing}
\begin{Verbatim}[commandchars=\\\{\}]
\PY{p}{]}\PY{n}{add}\PY{+w}{ }\PY{n}{Markdown}
\end{Verbatim}
\end{tcolorbox}

    \begin{Verbatim}[commandchars=\\\{\}]
\textcolor{ansi-green-intense}{\textbf{   Resolving}} package versions{\ldots}
\textcolor{ansi-green-intense}{\textbf{  No Changes}} to `\textasciitilde{}/MEGA/git-repos/infodev/Codes/Project.toml`
\textcolor{ansi-green-intense}{\textbf{  No Changes}} to `\textasciitilde{}/MEGA/git-
repos/infodev/Codes/Manifest.toml`
    \end{Verbatim}

    To be able to use it, we do as follows

    \begin{tcolorbox}[breakable, size=fbox, boxrule=1pt, pad at break*=1mm,colback=cellbackground, colframe=cellborder]
\prompt{In}{incolor}{4}{\boxspacing}
\begin{Verbatim}[commandchars=\\\{\}]
\PY{k}{using}\PY{+w}{ }\PY{n}{Markdown}
\end{Verbatim}
\end{tcolorbox}

    \begin{tcolorbox}[breakable, size=fbox, boxrule=1pt, pad at break*=1mm,colback=cellbackground, colframe=cellborder]
\prompt{In}{incolor}{5}{\boxspacing}
\begin{Verbatim}[commandchars=\\\{\}]
\PY{l+s+sa}{md}\PY{l+s}{\PYZdq{}\PYZdq{}\PYZdq{}}
\PY{l+s}{This a text inside a code cell, thanks to **Markdown** package. I can \PYZus{}emphasize\PYZus{} anything. Make other things **bold**}
\PY{l+s}{\PYZdq{}\PYZdq{}\PYZdq{}}
\end{Verbatim}
\end{tcolorbox}
 
            
\prompt{Out}{outcolor}{5}{}
    
    This a text inside a code cell, thanks to \textbf{Markdown} package. I can \emph{emphasize} anything. Make other things \textbf{bold}



    

    Runnig \texttt{Julia} in Jupyer Notebookor Jupyter Lab is pretty handy.
We only need to install the appropriate kernel. In order to add
\texttt{Julia} kernel \texttt{IJulia} to Jupyter Notebook and/or
\texttt{JupyterLab} IDEs, we begin by executing the following commands:

\begin{Shaded}
\begin{Highlighting}[]
\KeywordTok{using}\NormalTok{ Pkg  }
\NormalTok{Pkg.add(}\StringTok{"IJulia"}\NormalTok{)}
\end{Highlighting}
\end{Shaded}

If we want to get \texttt{JupyterLab} instance running in current
directory, we can do:

\begin{Shaded}
\begin{Highlighting}[]
\NormalTok{jupyterlab(dir}\OperatorTok{=}\NormalTok{pwd()}\OperatorTok{,}\NormalTok{ detached}\OperatorTok{=}\ExtensionTok{true}\NormalTok{)}
\end{Highlighting}
\end{Shaded}

    In case things do not work, we run the two following commands from
\({\tt Julia}\) REPL which launch jupyter environment.

\begin{Shaded}
\begin{Highlighting}[]
\KeywordTok{using}\NormalTok{ IJulia  }
\NormalTok{installkernel(}\StringTok{"Julia"}\NormalTok{)}
\end{Highlighting}
\end{Shaded}

    The shell mode is also available through the REPL to evaluate some os
commands. To do so, simply preface the regular command by semicolon. For
instance, \texttt{pwd} prints the path to working directory and
\texttt{ls} allows to list the content of the current directory.

    \begin{tcolorbox}[breakable, size=fbox, boxrule=1pt, pad at break*=1mm,colback=cellbackground, colframe=cellborder]
\prompt{In}{incolor}{6}{\boxspacing}
\begin{Verbatim}[commandchars=\\\{\}]
\PY{p}{;}\PY{n}{pwd}
\end{Verbatim}
\end{tcolorbox}

    \begin{Verbatim}[commandchars=\\\{\}]
/home/mhamdi/MEGA/git-repos/infodev/Codes
    \end{Verbatim}

    \begin{tcolorbox}[breakable, size=fbox, boxrule=1pt, pad at break*=1mm,colback=cellbackground, colframe=cellborder]
\prompt{In}{incolor}{7}{\boxspacing}
\begin{Verbatim}[commandchars=\\\{\}]
\PY{p}{;}\PY{n}{ls}\PY{+w}{ }\PY{o}{\PYZhy{}}\PY{n}{la}
\end{Verbatim}
\end{tcolorbox}

    \begin{Verbatim}[commandchars=\\\{\}]
total 752
drwxrwxr-x 3 mhamdi mhamdi   4096 Nov 29 21:29 .
drwxrwxr-x 6 mhamdi mhamdi   4096 Nov 19 16:27 ..
drwxrwxr-x 2 mhamdi mhamdi   4096 Nov 29 21:09 .ipynb\_checkpoints
-rw------- 3 mhamdi mhamdi 617224 Nov 29 21:29 julia-onramp.ipynb
-rw------- 1 mhamdi mhamdi  34286 Jan 14  2023 Julia.png
-rw-rw-r-- 1 mhamdi mhamdi  83778 Nov 29 21:13 Manifest.toml
-rw-rw-r-- 1 mhamdi mhamdi    444 Nov 29 21:13 Project.toml
-rw-rw-r-- 1 mhamdi mhamdi     20 Nov 19 16:27 README.md
-rw-rw-r-- 1 mhamdi mhamdi     66 Nov 29 21:28 test-file.csv
-rw------- 1 mhamdi mhamdi     18 Nov 29 21:17 .wakatime-project
    \end{Verbatim}

    \hypertarget{getting-help}{%
\section{Getting Help}\label{getting-help}}

In order to seek help on a particular function. We just use the
\texttt{?} mark. We can use the \({\tt Julia}\) documentation to
discover more pieces of information about \({\tt Julia}\) features.

    \begin{tcolorbox}[breakable, size=fbox, boxrule=1pt, pad at break*=1mm,colback=cellbackground, colframe=cellborder]
\prompt{In}{incolor}{8}{\boxspacing}
\begin{Verbatim}[commandchars=\\\{\}]
\PY{o}{?}\PY{n}{cos}
\end{Verbatim}
\end{tcolorbox}

    \begin{Verbatim}[commandchars=\\\{\}]
search: \textbf{c}\textbf{o}\textbf{s}
\textbf{c}\textbf{o}\textbf{s}h
\textbf{c}\textbf{o}\textbf{s}d
\textbf{c}\textbf{o}\textbf{s}c
\textbf{c}\textbf{o}\textbf{s}pi
a\textbf{c}\textbf{o}\textbf{s}
a\textbf{c}\textbf{o}\textbf{s}h
a\textbf{c}\textbf{o}\textbf{s}d
sin\textbf{c}\textbf{o}\textbf{s}
sin\textbf{c}\textbf{o}\textbf{s}d
sin\textbf{c}\textbf{o}\textbf{s}pi
\textbf{c}\textbf{o}n\textbf{s}t

    \end{Verbatim}
 
            
\prompt{Out}{outcolor}{8}{}
    
    \begin{verbatim}
cos(x)
\end{verbatim}
Compute cosine of \texttt{x}, where \texttt{x} is in radians.

See also \href{@ref}{\texttt{cosd}}, \href{@ref}{\texttt{cospi}}, \href{@ref}{\texttt{sincos}}, \href{@ref}{\texttt{cis}}.

\rule{\textwidth}{1pt}
\begin{verbatim}
cos(A::AbstractMatrix)
\end{verbatim}
Compute the matrix cosine of a square matrix \texttt{A}.

If \texttt{A} is symmetric or Hermitian, its eigendecomposition (\href{@ref}{\texttt{eigen}}) is used to compute the cosine. Otherwise, the cosine is determined by calling \href{@ref}{\texttt{exp}}.

\section{Examples}
\begin{verbatim}
julia> cos(fill(1.0, (2,2)))
2×2 Matrix{Float64}:
  0.291927  -0.708073
 -0.708073   0.291927
\end{verbatim}


    

    To print something on the standard output, it is possible to use either
\texttt{print} and \texttt{println}. The last one displays the text and
moves the cursor to the next line.

    \begin{tcolorbox}[breakable, size=fbox, boxrule=1pt, pad at break*=1mm,colback=cellbackground, colframe=cellborder]
\prompt{In}{incolor}{9}{\boxspacing}
\begin{Verbatim}[commandchars=\\\{\}]
\PY{n}{print}\PY{p}{(}\PY{l+s}{\PYZdq{}}\PY{l+s}{Hello}\PY{l+s}{\PYZdq{}}\PY{p}{)}
\PY{n}{print}\PY{p}{(}\PY{l+s+sc}{\PYZsq{} \PYZsq{}}\PY{p}{)}
\PY{n}{print}\PY{p}{(}\PY{l+s}{\PYZdq{}}\PY{l+s}{World}\PY{l+s}{\PYZdq{}}\PY{p}{)}
\end{Verbatim}
\end{tcolorbox}

    \begin{Verbatim}[commandchars=\\\{\}]
Hello World
    \end{Verbatim}

    \begin{tcolorbox}[breakable, size=fbox, boxrule=1pt, pad at break*=1mm,colback=cellbackground, colframe=cellborder]
\prompt{In}{incolor}{10}{\boxspacing}
\begin{Verbatim}[commandchars=\\\{\}]
\PY{n}{println}\PY{p}{(}\PY{l+s}{\PYZdq{}}\PY{l+s}{Hello}\PY{l+s}{\PYZdq{}}\PY{p}{)}
\PY{n}{println}\PY{p}{(}\PY{l+s}{\PYZdq{}}\PY{l+s}{World}\PY{l+s}{\PYZdq{}}\PY{p}{)}
\end{Verbatim}
\end{tcolorbox}

    \begin{Verbatim}[commandchars=\\\{\}]
Hello
World
    \end{Verbatim}

    \textbf{Data types: Dictionaries}

    \begin{tcolorbox}[breakable, size=fbox, boxrule=1pt, pad at break*=1mm,colback=cellbackground, colframe=cellborder]
\prompt{In}{incolor}{11}{\boxspacing}
\begin{Verbatim}[commandchars=\\\{\}]
\PY{n}{dict}\PY{+w}{ }\PY{o}{=}\PY{+w}{ }\PY{k+kt}{Dict}\PY{p}{(}
\PY{+w}{    }\PY{c}{\PYZsh{} Name =\PYZgt{} \PYZsh{} of wheels}
\PY{+w}{    }\PY{l+s}{\PYZdq{}}\PY{l+s}{Unicycle}\PY{l+s}{\PYZdq{}}\PY{+w}{ }\PY{o}{=\PYZgt{}}\PY{+w}{ }\PY{l+m+mi}{1}\PY{p}{,}
\PY{+w}{    }\PY{l+s}{\PYZdq{}}\PY{l+s}{Bicycle}\PY{l+s}{\PYZdq{}}\PY{+w}{ }\PY{o}{=\PYZgt{}}\PY{+w}{ }\PY{l+m+mi}{2}\PY{p}{,}
\PY{+w}{    }\PY{l+s}{\PYZdq{}}\PY{l+s}{Tricycle}\PY{l+s}{\PYZdq{}}\PY{+w}{ }\PY{o}{=\PYZgt{}}\PY{+w}{ }\PY{l+m+mi}{3}
\PY{p}{)}
\end{Verbatim}
\end{tcolorbox}

            \begin{tcolorbox}[breakable, size=fbox, boxrule=.5pt, pad at break*=1mm, opacityfill=0]
\prompt{Out}{outcolor}{11}{\boxspacing}
\begin{Verbatim}[commandchars=\\\{\}]
Dict\{String, Int64\} with 3 entries:
  "Bicycle"  => 2
  "Tricycle" => 3
  "Unicycle" => 1
\end{Verbatim}
\end{tcolorbox}
        
    \begin{tcolorbox}[breakable, size=fbox, boxrule=1pt, pad at break*=1mm,colback=cellbackground, colframe=cellborder]
\prompt{In}{incolor}{12}{\boxspacing}
\begin{Verbatim}[commandchars=\\\{\}]
\PY{n}{typeof}\PY{p}{(}\PY{n}{dict}\PY{p}{)}
\end{Verbatim}
\end{tcolorbox}

            \begin{tcolorbox}[breakable, size=fbox, boxrule=.5pt, pad at break*=1mm, opacityfill=0]
\prompt{Out}{outcolor}{12}{\boxspacing}
\begin{Verbatim}[commandchars=\\\{\}]
Dict\{String, Int64\}
\end{Verbatim}
\end{tcolorbox}
        
    \begin{tcolorbox}[breakable, size=fbox, boxrule=1pt, pad at break*=1mm,colback=cellbackground, colframe=cellborder]
\prompt{In}{incolor}{13}{\boxspacing}
\begin{Verbatim}[commandchars=\\\{\}]
\PY{n}{dict}\PY{+w}{ }\PY{o}{=}\PY{+w}{ }\PY{k+kt}{Dict}\PY{p}{(}\PY{p}{[}\PY{p}{(}\PY{l+s}{\PYZdq{}}\PY{l+s}{Unicycle}\PY{l+s}{\PYZdq{}}\PY{p}{,}\PY{+w}{ }\PY{l+m+mi}{1}\PY{p}{)}\PY{p}{,}\PY{+w}{ }\PY{p}{(}\PY{l+s}{\PYZdq{}}\PY{l+s}{Bicycle}\PY{l+s}{\PYZdq{}}\PY{p}{,}\PY{+w}{ }\PY{l+m+mi}{2}\PY{p}{)}\PY{p}{,}\PY{+w}{ }\PY{p}{(}\PY{l+s}{\PYZdq{}}\PY{l+s}{Tricycle}\PY{l+s}{\PYZdq{}}\PY{p}{,}\PY{+w}{ }\PY{l+m+mi}{3}\PY{p}{)}\PY{p}{]}\PY{p}{)}
\end{Verbatim}
\end{tcolorbox}

            \begin{tcolorbox}[breakable, size=fbox, boxrule=.5pt, pad at break*=1mm, opacityfill=0]
\prompt{Out}{outcolor}{13}{\boxspacing}
\begin{Verbatim}[commandchars=\\\{\}]
Dict\{String, Int64\} with 3 entries:
  "Bicycle"  => 2
  "Tricycle" => 3
  "Unicycle" => 1
\end{Verbatim}
\end{tcolorbox}
        
    \begin{tcolorbox}[breakable, size=fbox, boxrule=1pt, pad at break*=1mm,colback=cellbackground, colframe=cellborder]
\prompt{In}{incolor}{14}{\boxspacing}
\begin{Verbatim}[commandchars=\\\{\}]
\PY{n}{dict}\PY{p}{[}\PY{l+s}{\PYZdq{}}\PY{l+s}{Bicycle}\PY{l+s}{\PYZdq{}}\PY{p}{]}
\end{Verbatim}
\end{tcolorbox}

            \begin{tcolorbox}[breakable, size=fbox, boxrule=.5pt, pad at break*=1mm, opacityfill=0]
\prompt{Out}{outcolor}{14}{\boxspacing}
\begin{Verbatim}[commandchars=\\\{\}]
2
\end{Verbatim}
\end{tcolorbox}
        
    \begin{tcolorbox}[breakable, size=fbox, boxrule=1pt, pad at break*=1mm,colback=cellbackground, colframe=cellborder]
\prompt{In}{incolor}{15}{\boxspacing}
\begin{Verbatim}[commandchars=\\\{\}]
\PY{n}{lst}\PY{+w}{ }\PY{o}{=}\PY{+w}{ }\PY{p}{[}\PY{l+m+mi}{1}\PY{p}{,}\PY{+w}{ }\PY{l+s+sc}{\PYZsq{}a\PYZsq{}}\PY{p}{,}\PY{+w}{ }\PY{l+s}{\PYZdq{}}\PY{l+s}{abc}\PY{l+s}{\PYZdq{}}\PY{p}{,}\PY{+w}{ }\PY{n+nb}{true}\PY{p}{,}\PY{+w}{ }\PY{p}{[}\PY{l+m+mi}{0}\PY{p}{,}\PY{+w}{ }\PY{l+m+mf}{.5}\PY{n+nb}{im}\PY{p}{]}\PY{p}{]}
\end{Verbatim}
\end{tcolorbox}

            \begin{tcolorbox}[breakable, size=fbox, boxrule=.5pt, pad at break*=1mm, opacityfill=0]
\prompt{Out}{outcolor}{15}{\boxspacing}
\begin{Verbatim}[commandchars=\\\{\}]
5-element Vector\{Any\}:
    1
     'a': ASCII/Unicode U+0061 (category Ll: Letter, lowercase)
     "abc"
 true
     ComplexF64[0.0 + 0.0im, 0.0 + 0.5im]
\end{Verbatim}
\end{tcolorbox}
        
    \begin{tcolorbox}[breakable, size=fbox, boxrule=1pt, pad at break*=1mm,colback=cellbackground, colframe=cellborder]
\prompt{In}{incolor}{16}{\boxspacing}
\begin{Verbatim}[commandchars=\\\{\}]
\PY{n}{typeof}\PY{p}{(}\PY{n}{lst}\PY{p}{)}
\end{Verbatim}
\end{tcolorbox}

            \begin{tcolorbox}[breakable, size=fbox, boxrule=.5pt, pad at break*=1mm, opacityfill=0]
\prompt{Out}{outcolor}{16}{\boxspacing}
\begin{Verbatim}[commandchars=\\\{\}]
Vector\{Any\}\textcolor{ansi-black-intense}{ (alias for }\textcolor{ansi-black-intense}{Array\{Any, 1\}}\textcolor{ansi-black-intense}{)}
\end{Verbatim}
\end{tcolorbox}
        
    \begin{tcolorbox}[breakable, size=fbox, boxrule=1pt, pad at break*=1mm,colback=cellbackground, colframe=cellborder]
\prompt{In}{incolor}{17}{\boxspacing}
\begin{Verbatim}[commandchars=\\\{\}]
\PY{n}{lst}\PY{p}{[}\PY{k}{end}\PY{p}{]}
\end{Verbatim}
\end{tcolorbox}

            \begin{tcolorbox}[breakable, size=fbox, boxrule=.5pt, pad at break*=1mm, opacityfill=0]
\prompt{Out}{outcolor}{17}{\boxspacing}
\begin{Verbatim}[commandchars=\\\{\}]
2-element Vector\{ComplexF64\}:
 0.0 + 0.0im
 0.0 + 0.5im
\end{Verbatim}
\end{tcolorbox}
        
    \hypertarget{basic-calculations}{%
\subsubsection{Basic Calculations}\label{basic-calculations}}

    \begin{tcolorbox}[breakable, size=fbox, boxrule=1pt, pad at break*=1mm,colback=cellbackground, colframe=cellborder]
\prompt{In}{incolor}{18}{\boxspacing}
\begin{Verbatim}[commandchars=\\\{\}]
\PY{n}{a}\PY{p}{,}\PY{+w}{ }\PY{n}{b}\PY{+w}{ }\PY{o}{=}\PY{+w}{ }\PY{l+m+mi}{1}\PY{p}{,}\PY{+w}{ }\PY{l+m+mf}{1.5}
\end{Verbatim}
\end{tcolorbox}

            \begin{tcolorbox}[breakable, size=fbox, boxrule=.5pt, pad at break*=1mm, opacityfill=0]
\prompt{Out}{outcolor}{18}{\boxspacing}
\begin{Verbatim}[commandchars=\\\{\}]
(1, 1.5)
\end{Verbatim}
\end{tcolorbox}
        
    \begin{tcolorbox}[breakable, size=fbox, boxrule=1pt, pad at break*=1mm,colback=cellbackground, colframe=cellborder]
\prompt{In}{incolor}{19}{\boxspacing}
\begin{Verbatim}[commandchars=\\\{\}]
\PY{n}{println}\PY{p}{(}\PY{n}{typeof}\PY{p}{(}\PY{n}{a}\PY{p}{)}\PY{p}{)}
\PY{n}{println}\PY{p}{(}\PY{n}{typeof}\PY{p}{(}\PY{n}{b}\PY{p}{)}\PY{p}{)}
\end{Verbatim}
\end{tcolorbox}

    \begin{Verbatim}[commandchars=\\\{\}]
Int64
Float64
    \end{Verbatim}

    \begin{tcolorbox}[breakable, size=fbox, boxrule=1pt, pad at break*=1mm,colback=cellbackground, colframe=cellborder]
\prompt{In}{incolor}{20}{\boxspacing}
\begin{Verbatim}[commandchars=\\\{\}]
\PY{l+s+sa}{md}\PY{l+s}{\PYZdq{}\PYZdq{}\PYZdq{}}
\PY{l+s}{`varinfo` method allows to display loaded variables.}
\PY{l+s}{\PYZdq{}\PYZdq{}\PYZdq{}}
\end{Verbatim}
\end{tcolorbox}
 
            
\prompt{Out}{outcolor}{20}{}
    
    \texttt{varinfo} method allows to display loaded variables.



    

    \begin{tcolorbox}[breakable, size=fbox, boxrule=1pt, pad at break*=1mm,colback=cellbackground, colframe=cellborder]
\prompt{In}{incolor}{21}{\boxspacing}
\begin{Verbatim}[commandchars=\\\{\}]
\PY{n}{varinfo}\PY{p}{(}\PY{p}{)}
\end{Verbatim}
\end{tcolorbox}
 
            
\prompt{Out}{outcolor}{21}{}
    
    \begin{tabular}
{l | r | l}
name & size & summary \\
\hline
Base &  & Module \\
Core &  & Module \\
Main &  & Module \\
PLOTS\_DEFAULTS & 456 bytes & Dict\{Symbol, Symbol\} with 1 entry \\
a & 8 bytes & Int64 \\
b & 8 bytes & Float64 \\
dict & 503 bytes & Dict\{String, Int64\} with 3 entries \\
lst & 176 bytes & 5-element Vector\{Any\} \\
showall & 0 bytes & showall (generic function with 1 method) \\
\end{tabular}


    

    \begin{tcolorbox}[breakable, size=fbox, boxrule=1pt, pad at break*=1mm,colback=cellbackground, colframe=cellborder]
\prompt{In}{incolor}{22}{\boxspacing}
\begin{Verbatim}[commandchars=\\\{\}]
\PY{o}{?}\PY{n}{varinfo}
\end{Verbatim}
\end{tcolorbox}

    \begin{Verbatim}[commandchars=\\\{\}]
search: \textbf{v}\textbf{a}\textbf{r}\textbf{i}\textbf{n}[
0m\textbf{f}\textbf{o}

    \end{Verbatim}
 
            
\prompt{Out}{outcolor}{22}{}
    
    \begin{verbatim}
varinfo(m::Module=Main, pattern::Regex=r""; all::Bool = false, imported::Bool = false, sortby::Symbol = :name, minsize::Int = 0)
\end{verbatim}
Return a markdown table giving information about exported global variables in a module, optionally restricted to those matching \texttt{pattern}.

The memory consumption estimate is an approximate lower bound on the size of the internal structure of the object.

\begin{itemize}
\item \texttt{all} : also list non-exported objects defined in the module, deprecated objects, and compiler-generated objects.


\item \texttt{imported} : also list objects explicitly imported from other modules.


\item \texttt{recursive} : recursively include objects in sub-modules, observing the same settings in each.


\item \texttt{sortby} : the column to sort results by. Options are \texttt{:name} (default), \texttt{:size}, and \texttt{:summary}.


\item \texttt{minsize} : only includes objects with size at least \texttt{minsize} bytes. Defaults to \texttt{0}.

\end{itemize}


    

    \begin{tcolorbox}[breakable, size=fbox, boxrule=1pt, pad at break*=1mm,colback=cellbackground, colframe=cellborder]
\prompt{In}{incolor}{23}{\boxspacing}
\begin{Verbatim}[commandchars=\\\{\}]
\PY{n}{println}\PY{p}{(}\PY{l+s}{\PYZdq{}}\PY{l+s}{Sum of }\PY{l+s+si}{\PYZdl{}a}\PY{l+s}{ and }\PY{l+s+si}{\PYZdl{}b}\PY{l+s}{ is }\PY{l+s+si}{\PYZdl{}}\PY{p}{(}\PY{n}{a}\PY{o}{+}\PY{n}{b}\PY{p}{)}\PY{l+s}{\PYZdq{}}\PY{p}{)}
\end{Verbatim}
\end{tcolorbox}

    \begin{Verbatim}[commandchars=\\\{\}]
Sum of 1 and 1.5 is 2.5
    \end{Verbatim}

    Addition, subtraction, multiplication, division, exponent

    \begin{tcolorbox}[breakable, size=fbox, boxrule=1pt, pad at break*=1mm,colback=cellbackground, colframe=cellborder]
\prompt{In}{incolor}{24}{\boxspacing}
\begin{Verbatim}[commandchars=\\\{\}]
\PY{n}{a}\PY{o}{+}\PY{n}{b}\PY{p}{,}\PY{+w}{ }\PY{n}{a}\PY{o}{\PYZhy{}}\PY{n}{b}\PY{p}{,}\PY{+w}{ }\PY{n}{a}\PY{o}{*}\PY{n}{b}\PY{p}{,}\PY{+w}{ }\PY{n}{a÷b}\PY{p}{,}\PY{+w}{ }\PY{n}{a}\PY{o}{\PYZca{}}\PY{n}{b}
\end{Verbatim}
\end{tcolorbox}

            \begin{tcolorbox}[breakable, size=fbox, boxrule=.5pt, pad at break*=1mm, opacityfill=0]
\prompt{Out}{outcolor}{24}{\boxspacing}
\begin{Verbatim}[commandchars=\\\{\}]
(2.5, -0.5, 1.5, 0.0, 1.0)
\end{Verbatim}
\end{tcolorbox}
        
    \begin{tcolorbox}[breakable, size=fbox, boxrule=1pt, pad at break*=1mm,colback=cellbackground, colframe=cellborder]
\prompt{In}{incolor}{25}{\boxspacing}
\begin{Verbatim}[commandchars=\\\{\}]
\PY{l+s+sa}{md}\PY{l+s}{\PYZdq{}\PYZdq{}\PYZdq{}}
\PY{l+s}{**Unicode support** }
\PY{l+s}{1. We can use π instead of `pi`}
\PY{l+s}{1. Greek letters improe comprehension: α \PYZus{}(alpha)\PYZus{}, β \PYZus{}(beta)\PYZus{}, ...}
\PY{l+s}{1. Symbols: ≥ \PYZus{}(\PYZgt{}=)\PYZus{}, ≤ \PYZus{}(\PYZlt{}=)\PYZus{}, ∈ \PYZus{}(in)\PYZus{}, ... }
\PY{l+s}{1. ...}
\PY{l+s}{\PYZdq{}\PYZdq{}\PYZdq{}}
\end{Verbatim}
\end{tcolorbox}
 
            
\prompt{Out}{outcolor}{25}{}
    
    \textbf{Unicode support} 

\begin{itemize}
\item[1. ] We can use π instead of \texttt{pi}


\item[2. ] Greek letters improe comprehension: α \emph{(alpha)}, β \emph{(beta)}, ...


\item[3. ] Symbols: ≥ \emph{(>=)}, ≤ \emph{(<=)}, ∈ \emph{(in)}, ... 


\item[4. ] ...

\end{itemize}


    

    \begin{tcolorbox}[breakable, size=fbox, boxrule=1pt, pad at break*=1mm,colback=cellbackground, colframe=cellborder]
\prompt{In}{incolor}{26}{\boxspacing}
\begin{Verbatim}[commandchars=\\\{\}]
\PY{l+m+mi}{3}\PY{+w}{ }\PY{o}{≤}\PY{+w}{ }\PY{n+nb}{π}
\end{Verbatim}
\end{tcolorbox}

            \begin{tcolorbox}[breakable, size=fbox, boxrule=.5pt, pad at break*=1mm, opacityfill=0]
\prompt{Out}{outcolor}{26}{\boxspacing}
\begin{Verbatim}[commandchars=\\\{\}]
true
\end{Verbatim}
\end{tcolorbox}
        
    \begin{tcolorbox}[breakable, size=fbox, boxrule=1pt, pad at break*=1mm,colback=cellbackground, colframe=cellborder]
\prompt{In}{incolor}{27}{\boxspacing}
\begin{Verbatim}[commandchars=\\\{\}]
\PY{n}{typeof}\PY{p}{(}\PY{l+m+mf}{3.14}\PY{p}{)}
\end{Verbatim}
\end{tcolorbox}

            \begin{tcolorbox}[breakable, size=fbox, boxrule=.5pt, pad at break*=1mm, opacityfill=0]
\prompt{Out}{outcolor}{27}{\boxspacing}
\begin{Verbatim}[commandchars=\\\{\}]
Float64
\end{Verbatim}
\end{tcolorbox}
        
    \begin{tcolorbox}[breakable, size=fbox, boxrule=1pt, pad at break*=1mm,colback=cellbackground, colframe=cellborder]
\prompt{In}{incolor}{28}{\boxspacing}
\begin{Verbatim}[commandchars=\\\{\}]
\PY{k+kt}{Float64}\PY{+w}{ }\PY{o}{|\PYZgt{}}\PY{+w}{ }\PY{n}{supertype}\PY{+w}{ }\PY{o}{|\PYZgt{}}\PY{+w}{ }\PY{n}{supertype}\PY{+w}{ }\PY{o}{|\PYZgt{}}\PY{+w}{ }\PY{n}{supertype}\PY{+w}{ }\PY{o}{|\PYZgt{}}\PY{+w}{ }\PY{n}{supertype}
\end{Verbatim}
\end{tcolorbox}

            \begin{tcolorbox}[breakable, size=fbox, boxrule=.5pt, pad at break*=1mm, opacityfill=0]
\prompt{Out}{outcolor}{28}{\boxspacing}
\begin{Verbatim}[commandchars=\\\{\}]
Any
\end{Verbatim}
\end{tcolorbox}
        
    \begin{tcolorbox}[breakable, size=fbox, boxrule=1pt, pad at break*=1mm,colback=cellbackground, colframe=cellborder]
\prompt{In}{incolor}{29}{\boxspacing}
\begin{Verbatim}[commandchars=\\\{\}]
\PY{k+kt}{Integer}\PY{+w}{ }\PY{o}{|\PYZgt{}}\PY{+w}{ }\PY{n}{subtypes}
\end{Verbatim}
\end{tcolorbox}

            \begin{tcolorbox}[breakable, size=fbox, boxrule=.5pt, pad at break*=1mm, opacityfill=0]
\prompt{Out}{outcolor}{29}{\boxspacing}
\begin{Verbatim}[commandchars=\\\{\}]
3-element Vector\{Any\}:
 Bool
 Signed
 Unsigned
\end{Verbatim}
\end{tcolorbox}
        
    \begin{tcolorbox}[breakable, size=fbox, boxrule=1pt, pad at break*=1mm,colback=cellbackground, colframe=cellborder]
\prompt{In}{incolor}{30}{\boxspacing}
\begin{Verbatim}[commandchars=\\\{\}]
\PY{k+kt}{Signed}\PY{+w}{ }\PY{o}{|\PYZgt{}}\PY{+w}{ }\PY{n}{subtypes}
\end{Verbatim}
\end{tcolorbox}

            \begin{tcolorbox}[breakable, size=fbox, boxrule=.5pt, pad at break*=1mm, opacityfill=0]
\prompt{Out}{outcolor}{30}{\boxspacing}
\begin{Verbatim}[commandchars=\\\{\}]
6-element Vector\{Any\}:
 BigInt
 Int128
 Int16
 Int32
 Int64
 Int8
\end{Verbatim}
\end{tcolorbox}
        
    \begin{tcolorbox}[breakable, size=fbox, boxrule=1pt, pad at break*=1mm,colback=cellbackground, colframe=cellborder]
\prompt{In}{incolor}{31}{\boxspacing}
\begin{Verbatim}[commandchars=\\\{\}]
\PY{k+kt}{UInt8}\PY{+w}{ }\PY{o}{\PYZlt{}:}\PY{+w}{ }\PY{k+kt}{Unsigned}\PY{+w}{ }\PY{c}{\PYZsh{} UInt8 is one subtype of Unsigned}
\end{Verbatim}
\end{tcolorbox}

            \begin{tcolorbox}[breakable, size=fbox, boxrule=.5pt, pad at break*=1mm, opacityfill=0]
\prompt{Out}{outcolor}{31}{\boxspacing}
\begin{Verbatim}[commandchars=\\\{\}]
true
\end{Verbatim}
\end{tcolorbox}
        
    \begin{tcolorbox}[breakable, size=fbox, boxrule=1pt, pad at break*=1mm,colback=cellbackground, colframe=cellborder]
\prompt{In}{incolor}{32}{\boxspacing}
\begin{Verbatim}[commandchars=\\\{\}]
\PY{k+kt}{Signed}\PY{+w}{ }\PY{o}{\PYZgt{}:}\PY{+w}{ }\PY{k+kt}{Int8}\PY{+w}{ }\PY{c}{\PYZsh{} Signed is supertype of Int8}
\end{Verbatim}
\end{tcolorbox}

            \begin{tcolorbox}[breakable, size=fbox, boxrule=.5pt, pad at break*=1mm, opacityfill=0]
\prompt{Out}{outcolor}{32}{\boxspacing}
\begin{Verbatim}[commandchars=\\\{\}]
true
\end{Verbatim}
\end{tcolorbox}
        
    \begin{tcolorbox}[breakable, size=fbox, boxrule=1pt, pad at break*=1mm,colback=cellbackground, colframe=cellborder]
\prompt{In}{incolor}{33}{\boxspacing}
\begin{Verbatim}[commandchars=\\\{\}]
\PY{n}{typeof}\PY{p}{(}\PY{l+m+mi}{3}\PY{p}{)}
\end{Verbatim}
\end{tcolorbox}

            \begin{tcolorbox}[breakable, size=fbox, boxrule=.5pt, pad at break*=1mm, opacityfill=0]
\prompt{Out}{outcolor}{33}{\boxspacing}
\begin{Verbatim}[commandchars=\\\{\}]
Int64
\end{Verbatim}
\end{tcolorbox}
        
    \begin{tcolorbox}[breakable, size=fbox, boxrule=1pt, pad at break*=1mm,colback=cellbackground, colframe=cellborder]
\prompt{In}{incolor}{34}{\boxspacing}
\begin{Verbatim}[commandchars=\\\{\}]
\PY{n}{tmp}\PY{o}{::}\PY{k+kt}{UInt8}\PY{+w}{ }\PY{o}{=}\PY{+w}{ }\PY{l+m+mi}{3}
\PY{n}{typeof}\PY{p}{(}\PY{n}{tmp}\PY{p}{)}
\end{Verbatim}
\end{tcolorbox}

            \begin{tcolorbox}[breakable, size=fbox, boxrule=.5pt, pad at break*=1mm, opacityfill=0]
\prompt{Out}{outcolor}{34}{\boxspacing}
\begin{Verbatim}[commandchars=\\\{\}]
UInt8
\end{Verbatim}
\end{tcolorbox}
        
    \hypertarget{mathematical-notation}{%
\subsubsection{Mathematical Notation}\label{mathematical-notation}}

    \begin{tcolorbox}[breakable, size=fbox, boxrule=1pt, pad at break*=1mm,colback=cellbackground, colframe=cellborder]
\prompt{In}{incolor}{35}{\boxspacing}
\begin{Verbatim}[commandchars=\\\{\}]
\PY{n}{println}\PY{p}{(}\PY{l+m+mi}{1}\PY{o}{+}\PY{l+m+mi}{2}\PY{p}{)}
\PY{n}{println}\PY{p}{(}\PY{o}{+}\PY{p}{(}\PY{l+m+mi}{1}\PY{p}{,}\PY{+w}{ }\PY{l+m+mi}{2}\PY{p}{)}\PY{p}{)}
\end{Verbatim}
\end{tcolorbox}

    \begin{Verbatim}[commandchars=\\\{\}]
3
3
    \end{Verbatim}

    \begin{tcolorbox}[breakable, size=fbox, boxrule=1pt, pad at break*=1mm,colback=cellbackground, colframe=cellborder]
\prompt{In}{incolor}{36}{\boxspacing}
\begin{Verbatim}[commandchars=\\\{\}]
\PY{n}{println}\PY{p}{(}\PY{l+m+mi}{1}\PY{o}{\PYZhy{}}\PY{l+m+mi}{2}\PY{p}{)}
\PY{n}{println}\PY{p}{(}\PY{o}{\PYZhy{}}\PY{p}{(}\PY{l+m+mi}{1}\PY{p}{,}\PY{+w}{ }\PY{l+m+mi}{2}\PY{p}{)}\PY{p}{)}
\end{Verbatim}
\end{tcolorbox}

    \begin{Verbatim}[commandchars=\\\{\}]
-1
-1
    \end{Verbatim}

    \begin{tcolorbox}[breakable, size=fbox, boxrule=1pt, pad at break*=1mm,colback=cellbackground, colframe=cellborder]
\prompt{In}{incolor}{37}{\boxspacing}
\begin{Verbatim}[commandchars=\\\{\}]
\PY{n}{println}\PY{p}{(}\PY{l+m+mi}{1}\PY{o}{*}\PY{l+m+mi}{2}\PY{p}{)}
\PY{n}{println}\PY{p}{(}\PY{o}{*}\PY{p}{(}\PY{l+m+mi}{1}\PY{p}{,}\PY{+w}{ }\PY{l+m+mi}{2}\PY{p}{)}\PY{p}{)}
\end{Verbatim}
\end{tcolorbox}

    \begin{Verbatim}[commandchars=\\\{\}]
2
2
    \end{Verbatim}

    \begin{tcolorbox}[breakable, size=fbox, boxrule=1pt, pad at break*=1mm,colback=cellbackground, colframe=cellborder]
\prompt{In}{incolor}{38}{\boxspacing}
\begin{Verbatim}[commandchars=\\\{\}]
\PY{n}{println}\PY{p}{(}\PY{l+m+mi}{1}\PY{o}{/}\PY{l+m+mi}{2}\PY{p}{)}
\PY{n}{println}\PY{p}{(}\PY{o}{/}\PY{p}{(}\PY{l+m+mi}{1}\PY{p}{,}\PY{+w}{ }\PY{l+m+mi}{2}\PY{p}{)}\PY{p}{)}
\end{Verbatim}
\end{tcolorbox}

    \begin{Verbatim}[commandchars=\\\{\}]
0.5
0.5
    \end{Verbatim}

    \begin{tcolorbox}[breakable, size=fbox, boxrule=1pt, pad at break*=1mm,colback=cellbackground, colframe=cellborder]
\prompt{In}{incolor}{39}{\boxspacing}
\begin{Verbatim}[commandchars=\\\{\}]
\PY{n}{println}\PY{p}{(}\PY{l+m+mi}{3}\PY{o}{/}\PY{l+m+mi}{4}\PY{o}{+}\PY{l+m+mi}{7}\PY{o}{/}\PY{l+m+mi}{5}\PY{p}{)}
\PY{n}{println}\PY{p}{(}\PY{l+m+mi}{3}\PY{o}{//}\PY{l+m+mi}{4}\PY{o}{+}\PY{l+m+mi}{7}\PY{o}{//}\PY{l+m+mi}{5}\PY{p}{)}
\end{Verbatim}
\end{tcolorbox}

    \begin{Verbatim}[commandchars=\\\{\}]
2.15
43//20
    \end{Verbatim}

    \hypertarget{array-transformations}{%
\subsubsection{Array Transformations}\label{array-transformations}}

Perform calculations on entire arrays at once.

    \begin{tcolorbox}[breakable, size=fbox, boxrule=1pt, pad at break*=1mm,colback=cellbackground, colframe=cellborder]
\prompt{In}{incolor}{40}{\boxspacing}
\begin{Verbatim}[commandchars=\\\{\}]
\PY{n}{zeros}\PY{p}{(}\PY{l+m+mi}{3}\PY{p}{,}\PY{+w}{ }\PY{l+m+mi}{2}\PY{p}{)}
\end{Verbatim}
\end{tcolorbox}

            \begin{tcolorbox}[breakable, size=fbox, boxrule=.5pt, pad at break*=1mm, opacityfill=0]
\prompt{Out}{outcolor}{40}{\boxspacing}
\begin{Verbatim}[commandchars=\\\{\}]
3×2 Matrix\{Float64\}:
 0.0  0.0
 0.0  0.0
 0.0  0.0
\end{Verbatim}
\end{tcolorbox}
        
    \begin{tcolorbox}[breakable, size=fbox, boxrule=1pt, pad at break*=1mm,colback=cellbackground, colframe=cellborder]
\prompt{In}{incolor}{41}{\boxspacing}
\begin{Verbatim}[commandchars=\\\{\}]
\PY{n}{ones}\PY{p}{(}\PY{l+m+mi}{3}\PY{p}{,}\PY{+w}{ }\PY{l+m+mi}{3}\PY{p}{,}\PY{+w}{ }\PY{l+m+mi}{2}\PY{p}{)}
\end{Verbatim}
\end{tcolorbox}

            \begin{tcolorbox}[breakable, size=fbox, boxrule=.5pt, pad at break*=1mm, opacityfill=0]
\prompt{Out}{outcolor}{41}{\boxspacing}
\begin{Verbatim}[commandchars=\\\{\}]
3×3×2 Array\{Float64, 3\}:
[:, :, 1] =
 1.0  1.0  1.0
 1.0  1.0  1.0
 1.0  1.0  1.0

[:, :, 2] =
 1.0  1.0  1.0
 1.0  1.0  1.0
 1.0  1.0  1.0
\end{Verbatim}
\end{tcolorbox}
        
    \begin{tcolorbox}[breakable, size=fbox, boxrule=1pt, pad at break*=1mm,colback=cellbackground, colframe=cellborder]
\prompt{In}{incolor}{42}{\boxspacing}
\begin{Verbatim}[commandchars=\\\{\}]
\PY{n}{fill}\PY{p}{(}\PY{n+nb}{π}\PY{p}{,}\PY{+w}{ }\PY{p}{(}\PY{l+m+mi}{2}\PY{p}{,}\PY{+w}{ }\PY{l+m+mi}{2}\PY{p}{)}\PY{p}{)}
\end{Verbatim}
\end{tcolorbox}

            \begin{tcolorbox}[breakable, size=fbox, boxrule=.5pt, pad at break*=1mm, opacityfill=0]
\prompt{Out}{outcolor}{42}{\boxspacing}
\begin{Verbatim}[commandchars=\\\{\}]
2×2 Matrix\{Irrational\{:π\}\}:
 π  π
 π  π
\end{Verbatim}
\end{tcolorbox}
        
    \begin{tcolorbox}[breakable, size=fbox, boxrule=1pt, pad at break*=1mm,colback=cellbackground, colframe=cellborder]
\prompt{In}{incolor}{43}{\boxspacing}
\begin{Verbatim}[commandchars=\\\{\}]
\PY{n+nb}{π}\PY{+w}{ }\PY{o}{.*}\PY{+w}{ }\PY{n}{ones}\PY{p}{(}\PY{l+m+mi}{2}\PY{p}{,}\PY{+w}{ }\PY{l+m+mi}{2}\PY{p}{)}
\end{Verbatim}
\end{tcolorbox}

            \begin{tcolorbox}[breakable, size=fbox, boxrule=.5pt, pad at break*=1mm, opacityfill=0]
\prompt{Out}{outcolor}{43}{\boxspacing}
\begin{Verbatim}[commandchars=\\\{\}]
2×2 Matrix\{Float64\}:
 3.14159  3.14159
 3.14159  3.14159
\end{Verbatim}
\end{tcolorbox}
        
    \begin{tcolorbox}[breakable, size=fbox, boxrule=1pt, pad at break*=1mm,colback=cellbackground, colframe=cellborder]
\prompt{In}{incolor}{44}{\boxspacing}
\begin{Verbatim}[commandchars=\\\{\}]
\PY{l+s+sa}{md}\PY{l+s}{\PYZdq{}}\PY{l+s}{Creates a `BitArray` with all values set to `true`}\PY{l+s}{\PYZdq{}}
\end{Verbatim}
\end{tcolorbox}
 
            
\prompt{Out}{outcolor}{44}{}
    
    Creates a \texttt{BitArray} with all values set to \texttt{true}



    

    \begin{tcolorbox}[breakable, size=fbox, boxrule=1pt, pad at break*=1mm,colback=cellbackground, colframe=cellborder]
\prompt{In}{incolor}{45}{\boxspacing}
\begin{Verbatim}[commandchars=\\\{\}]
\PY{n}{var}\PY{+w}{ }\PY{o}{=}\PY{+w}{ }\PY{n}{trues}\PY{p}{(}\PY{l+m+mi}{2}\PY{p}{,}\PY{+w}{ }\PY{l+m+mi}{4}\PY{p}{)}
\PY{n}{println}\PY{p}{(}\PY{n}{var}\PY{p}{)}
\PY{n}{typeof}\PY{p}{(}\PY{n}{var}\PY{p}{)}
\end{Verbatim}
\end{tcolorbox}

    \begin{Verbatim}[commandchars=\\\{\}]
Bool[1 1 1 1; 1 1 1 1]
    \end{Verbatim}

            \begin{tcolorbox}[breakable, size=fbox, boxrule=.5pt, pad at break*=1mm, opacityfill=0]
\prompt{Out}{outcolor}{45}{\boxspacing}
\begin{Verbatim}[commandchars=\\\{\}]
BitMatrix\textcolor{ansi-black-intense}{ (alias for }\textcolor{ansi-black-intense}{BitArray\{2\}}\textcolor{ansi-black-intense}{)}
\end{Verbatim}
\end{tcolorbox}
        
    \begin{tcolorbox}[breakable, size=fbox, boxrule=1pt, pad at break*=1mm,colback=cellbackground, colframe=cellborder]
\prompt{In}{incolor}{46}{\boxspacing}
\begin{Verbatim}[commandchars=\\\{\}]
\PY{l+s+sa}{md}\PY{l+s}{\PYZdq{}}\PY{l+s}{Creates a `BitArray` with all values set to `false`}\PY{l+s}{\PYZdq{}}
\end{Verbatim}
\end{tcolorbox}
 
            
\prompt{Out}{outcolor}{46}{}
    
    Creates a \texttt{BitArray} with all values set to \texttt{false}



    

    \begin{tcolorbox}[breakable, size=fbox, boxrule=1pt, pad at break*=1mm,colback=cellbackground, colframe=cellborder]
\prompt{In}{incolor}{47}{\boxspacing}
\begin{Verbatim}[commandchars=\\\{\}]
\PY{n}{var}\PY{+w}{ }\PY{o}{=}\PY{+w}{ }\PY{n}{falses}\PY{p}{(}\PY{l+m+mi}{2}\PY{p}{,}\PY{+w}{ }\PY{l+m+mi}{4}\PY{p}{)}
\PY{n}{println}\PY{p}{(}\PY{n}{var}\PY{p}{)}
\PY{n}{typeof}\PY{p}{(}\PY{n}{var}\PY{p}{)}
\end{Verbatim}
\end{tcolorbox}

    \begin{Verbatim}[commandchars=\\\{\}]
Bool[0 0 0 0; 0 0 0 0]
    \end{Verbatim}

            \begin{tcolorbox}[breakable, size=fbox, boxrule=.5pt, pad at break*=1mm, opacityfill=0]
\prompt{Out}{outcolor}{47}{\boxspacing}
\begin{Verbatim}[commandchars=\\\{\}]
BitMatrix\textcolor{ansi-black-intense}{ (alias for }\textcolor{ansi-black-intense}{BitArray\{2\}}\textcolor{ansi-black-intense}{)}
\end{Verbatim}
\end{tcolorbox}
        
    \begin{tcolorbox}[breakable, size=fbox, boxrule=1pt, pad at break*=1mm,colback=cellbackground, colframe=cellborder]
\prompt{In}{incolor}{48}{\boxspacing}
\begin{Verbatim}[commandchars=\\\{\}]
\PY{l+s+sa}{md}\PY{l+s}{\PYZdq{}}\PY{l+s}{**Comprehension**}\PY{l+s}{\PYZdq{}}
\end{Verbatim}
\end{tcolorbox}
 
            
\prompt{Out}{outcolor}{48}{}
    
    \textbf{Comprehension}



    

    \begin{tcolorbox}[breakable, size=fbox, boxrule=1pt, pad at break*=1mm,colback=cellbackground, colframe=cellborder]
\prompt{In}{incolor}{49}{\boxspacing}
\begin{Verbatim}[commandchars=\\\{\}]
\PY{n}{str}\PY{+w}{ }\PY{o}{=}\PY{+w}{ }\PY{l+s}{\PYZdq{}}\PY{l+s}{Hello Julia}\PY{l+s}{\PYZdq{}}
\PY{p}{[}\PY{n}{println}\PY{p}{(}\PY{n}{el}\PY{p}{)}\PY{+w}{ }\PY{k}{for}\PY{+w}{ }\PY{n}{el}\PY{+w}{ }\PY{k}{in}\PY{+w}{ }\PY{n}{str}\PY{p}{]}\PY{p}{;}
\end{Verbatim}
\end{tcolorbox}

    \begin{Verbatim}[commandchars=\\\{\}]
H
e
l
l
o

J
u
l
i
a
    \end{Verbatim}

    \hypertarget{calling-functions}{%
\subsubsection{Calling Functions}\label{calling-functions}}

Call functions to obtain multiple outputs.

    \begin{tcolorbox}[breakable, size=fbox, boxrule=1pt, pad at break*=1mm,colback=cellbackground, colframe=cellborder]
\prompt{In}{incolor}{50}{\boxspacing}
\begin{Verbatim}[commandchars=\\\{\}]
\PY{l+s+sa}{md}\PY{l+s}{\PYZdq{}\PYZdq{}\PYZdq{}}
\PY{l+s}{[Functions in }\PY{l+s}{\PYZdl{}}\PY{l+s}{\PYZob{}}\PY{l+s+se}{\PYZbs{}t}\PY{l+s}{t Julia\PYZcb{}}\PY{l+s}{\PYZdl{}}\PY{l+s}{](https://docs.julialang.org/en/v1/manual/functions/)}
\PY{l+s}{\PYZdq{}\PYZdq{}\PYZdq{}}
\end{Verbatim}
\end{tcolorbox}
 
            
\prompt{Out}{outcolor}{50}{}
    
    \href{https://docs.julialang.org/en/v1/manual/functions/}{Functions in ${\tt Julia}$}



    

    \begin{tcolorbox}[breakable, size=fbox, boxrule=1pt, pad at break*=1mm,colback=cellbackground, colframe=cellborder]
\prompt{In}{incolor}{51}{\boxspacing}
\begin{Verbatim}[commandchars=\\\{\}]
\PY{l+s+sa}{md}\PY{l+s}{\PYZdq{}}\PY{l+s}{**Spreading Arguments**}\PY{l+s}{\PYZdq{}}
\end{Verbatim}
\end{tcolorbox}
 
            
\prompt{Out}{outcolor}{51}{}
    
    \textbf{Spreading Arguments}



    

    \emph{Optional positional arguments}

    \begin{tcolorbox}[breakable, size=fbox, boxrule=1pt, pad at break*=1mm,colback=cellbackground, colframe=cellborder]
\prompt{In}{incolor}{52}{\boxspacing}
\begin{Verbatim}[commandchars=\\\{\}]
\PY{n}{foo}\PY{p}{(}\PY{n}{x}\PY{o}{=}\PY{l+m+mi}{0}\PY{p}{,}\PY{+w}{ }\PY{n}{y}\PY{o}{=}\PY{l+m+mi}{0}\PY{p}{,}\PY{+w}{ }\PY{n}{z}\PY{o}{=}\PY{l+m+mi}{0}\PY{p}{)}\PY{+w}{ }\PY{o}{=}\PY{+w}{ }\PY{n}{x}\PY{o}{+}\PY{n}{y}\PY{o}{+}\PY{n}{z}
\end{Verbatim}
\end{tcolorbox}

            \begin{tcolorbox}[breakable, size=fbox, boxrule=.5pt, pad at break*=1mm, opacityfill=0]
\prompt{Out}{outcolor}{52}{\boxspacing}
\begin{Verbatim}[commandchars=\\\{\}]
foo (generic function with 4 methods)
\end{Verbatim}
\end{tcolorbox}
        
    \begin{tcolorbox}[breakable, size=fbox, boxrule=1pt, pad at break*=1mm,colback=cellbackground, colframe=cellborder]
\prompt{In}{incolor}{53}{\boxspacing}
\begin{Verbatim}[commandchars=\\\{\}]
\PY{n}{foo}\PY{p}{(}\PY{p}{)}\PY{p}{,}\PY{+w}{ }\PY{n}{foo}\PY{p}{(}\PY{l+m+mi}{1}\PY{p}{,}\PY{+w}{ }\PY{l+m+mi}{2}\PY{p}{,}\PY{+w}{ }\PY{l+m+mi}{3}\PY{p}{)}
\end{Verbatim}
\end{tcolorbox}

            \begin{tcolorbox}[breakable, size=fbox, boxrule=.5pt, pad at break*=1mm, opacityfill=0]
\prompt{Out}{outcolor}{53}{\boxspacing}
\begin{Verbatim}[commandchars=\\\{\}]
(0, 6)
\end{Verbatim}
\end{tcolorbox}
        
    \begin{tcolorbox}[breakable, size=fbox, boxrule=1pt, pad at break*=1mm,colback=cellbackground, colframe=cellborder]
\prompt{In}{incolor}{54}{\boxspacing}
\begin{Verbatim}[commandchars=\\\{\}]
\PY{n}{foo}\PY{p}{(}\PY{p}{[}\PY{l+m+mi}{1}\PY{p}{,}\PY{+w}{ }\PY{l+m+mi}{2}\PY{p}{,}\PY{+w}{ }\PY{l+m+mi}{3}\PY{p}{]}\PY{o}{...}\PY{p}{)}\PY{+w}{ }\PY{c}{\PYZsh{} Splat `...` operator}
\end{Verbatim}
\end{tcolorbox}

            \begin{tcolorbox}[breakable, size=fbox, boxrule=.5pt, pad at break*=1mm, opacityfill=0]
\prompt{Out}{outcolor}{54}{\boxspacing}
\begin{Verbatim}[commandchars=\\\{\}]
6
\end{Verbatim}
\end{tcolorbox}
        
    \emph{Keywords arguments}

    \begin{tcolorbox}[breakable, size=fbox, boxrule=1pt, pad at break*=1mm,colback=cellbackground, colframe=cellborder]
\prompt{In}{incolor}{55}{\boxspacing}
\begin{Verbatim}[commandchars=\\\{\}]
\PY{n}{bar}\PY{p}{(}\PY{p}{;}\PY{+w}{ }\PY{n}{a}\PY{o}{::}\PY{k+kt}{Real}\PY{o}{=}\PY{l+m+mi}{0}\PY{p}{,}\PY{+w}{ }\PY{n}{b}\PY{o}{::}\PY{k+kt}{Real}\PY{o}{=}\PY{l+m+mi}{0}\PY{p}{,}\PY{+w}{ }\PY{n}{c}\PY{o}{::}\PY{k+kt}{Real}\PY{o}{=}\PY{l+m+mi}{0}\PY{p}{)}\PY{+w}{ }\PY{o}{=}\PY{+w}{ }\PY{n}{a}\PY{o}{+}\PY{n}{b}\PY{o}{+}\PY{n}{c}
\end{Verbatim}
\end{tcolorbox}

            \begin{tcolorbox}[breakable, size=fbox, boxrule=.5pt, pad at break*=1mm, opacityfill=0]
\prompt{Out}{outcolor}{55}{\boxspacing}
\begin{Verbatim}[commandchars=\\\{\}]
bar (generic function with 1 method)
\end{Verbatim}
\end{tcolorbox}
        
    \begin{tcolorbox}[breakable, size=fbox, boxrule=1pt, pad at break*=1mm,colback=cellbackground, colframe=cellborder]
\prompt{In}{incolor}{56}{\boxspacing}
\begin{Verbatim}[commandchars=\\\{\}]
\PY{n}{bar}\PY{p}{(}\PY{p}{)}
\end{Verbatim}
\end{tcolorbox}

            \begin{tcolorbox}[breakable, size=fbox, boxrule=.5pt, pad at break*=1mm, opacityfill=0]
\prompt{Out}{outcolor}{56}{\boxspacing}
\begin{Verbatim}[commandchars=\\\{\}]
0
\end{Verbatim}
\end{tcolorbox}
        
    \begin{tcolorbox}[breakable, size=fbox, boxrule=1pt, pad at break*=1mm,colback=cellbackground, colframe=cellborder]
\prompt{In}{incolor}{57}{\boxspacing}
\begin{Verbatim}[commandchars=\\\{\}]
\PY{n}{bar}\PY{p}{(}\PY{p}{;}\PY{+w}{ }\PY{k+kt}{Dict}\PY{p}{(}\PY{l+s+ss}{:a}\PY{+w}{ }\PY{o}{=\PYZgt{}}\PY{+w}{ }\PY{l+m+mi}{3}\PY{p}{,}\PY{+w}{ }\PY{l+s+ss}{:b}\PY{+w}{ }\PY{o}{=\PYZgt{}}\PY{+w}{ }\PY{l+m+mf}{5.4}\PY{p}{,}\PY{+w}{ }\PY{l+s+ss}{:c}\PY{+w}{ }\PY{o}{=\PYZgt{}}\PY{+w}{ }\PY{o}{\PYZhy{}}\PY{l+m+mf}{1.2}\PY{p}{)}\PY{o}{...}\PY{+w}{ }\PY{p}{)}\PY{+w}{ }\PY{c}{\PYZsh{} ; kwargs...}
\end{Verbatim}
\end{tcolorbox}

            \begin{tcolorbox}[breakable, size=fbox, boxrule=.5pt, pad at break*=1mm, opacityfill=0]
\prompt{Out}{outcolor}{57}{\boxspacing}
\begin{Verbatim}[commandchars=\\\{\}]
7.2
\end{Verbatim}
\end{tcolorbox}
        
    \begin{tcolorbox}[breakable, size=fbox, boxrule=1pt, pad at break*=1mm,colback=cellbackground, colframe=cellborder]
\prompt{In}{incolor}{58}{\boxspacing}
\begin{Verbatim}[commandchars=\\\{\}]
\PY{c}{\PYZsh{} THROW AN ERROR }
\PY{k}{try}\PY{+w}{ }\PY{n}{bar}\PY{p}{(}\PY{p}{[}\PY{l+m+mi}{1}\PY{p}{,}\PY{+w}{ }\PY{l+m+mi}{2}\PY{p}{,}\PY{+w}{ }\PY{l+m+mi}{3}\PY{p}{]}\PY{o}{...}\PY{p}{)}
\PY{k}{catch}\PY{+w}{ }\PY{n}{error}
\PY{+w}{    }\PY{n}{println}\PY{p}{(}\PY{n}{error}\PY{p}{)}
\PY{k}{end}
\end{Verbatim}
\end{tcolorbox}

    \begin{Verbatim}[commandchars=\\\{\}]
MethodError(bar, (1, 2, 3), 0x00000000000082db)
    \end{Verbatim}

    \begin{tcolorbox}[breakable, size=fbox, boxrule=1pt, pad at break*=1mm,colback=cellbackground, colframe=cellborder]
\prompt{In}{incolor}{59}{\boxspacing}
\begin{Verbatim}[commandchars=\\\{\}]
\PY{l+s+sa}{md}\PY{l+s}{\PYZdq{}}\PY{l+s}{**Multiple Dispatch**}\PY{l+s}{\PYZdq{}}
\end{Verbatim}
\end{tcolorbox}
 
            
\prompt{Out}{outcolor}{59}{}
    
    \textbf{Multiple Dispatch}



    

    \begin{tcolorbox}[breakable, size=fbox, boxrule=1pt, pad at break*=1mm,colback=cellbackground, colframe=cellborder]
\prompt{In}{incolor}{60}{\boxspacing}
\begin{Verbatim}[commandchars=\\\{\}]
\PY{c}{\PYZsh{} 1st method signature}
\PY{k}{function}\PY{+w}{ }\PY{n}{f}\PY{p}{(}\PY{n}{x}\PY{o}{::}\PY{k+kt}{Int}\PY{p}{)}
\PY{+w}{    }\PY{n}{x}\PY{o}{\PYZca{}}\PY{l+m+mi}{2}
\PY{k}{end}
\end{Verbatim}
\end{tcolorbox}

            \begin{tcolorbox}[breakable, size=fbox, boxrule=.5pt, pad at break*=1mm, opacityfill=0]
\prompt{Out}{outcolor}{60}{\boxspacing}
\begin{Verbatim}[commandchars=\\\{\}]
f (generic function with 1 method)
\end{Verbatim}
\end{tcolorbox}
        
    \begin{tcolorbox}[breakable, size=fbox, boxrule=1pt, pad at break*=1mm,colback=cellbackground, colframe=cellborder]
\prompt{In}{incolor}{61}{\boxspacing}
\begin{Verbatim}[commandchars=\\\{\}]
\PY{c}{\PYZsh{} 2nd method signature}
\PY{n}{f}\PY{p}{(}\PY{n}{x}\PY{o}{::}\PY{k+kt}{Float64}\PY{p}{)}\PY{+w}{ }\PY{o}{=}\PY{+w}{ }\PY{n}{x}\PY{o}{\PYZca{}}\PY{l+m+mi}{2}\PY{o}{+}\PY{l+m+mi}{1}
\end{Verbatim}
\end{tcolorbox}

            \begin{tcolorbox}[breakable, size=fbox, boxrule=.5pt, pad at break*=1mm, opacityfill=0]
\prompt{Out}{outcolor}{61}{\boxspacing}
\begin{Verbatim}[commandchars=\\\{\}]
f (generic function with 2 methods)
\end{Verbatim}
\end{tcolorbox}
        
    \begin{tcolorbox}[breakable, size=fbox, boxrule=1pt, pad at break*=1mm,colback=cellbackground, colframe=cellborder]
\prompt{In}{incolor}{62}{\boxspacing}
\begin{Verbatim}[commandchars=\\\{\}]
\PY{c}{\PYZsh{} 3rd method signature}
\PY{n}{f}\PY{p}{(}\PY{n}{x}\PY{o}{::}\PY{k+kt}{Char}\PY{p}{)}\PY{+w}{ }\PY{o}{=}\PY{+w}{ }\PY{n}{x}\PY{o}{*}\PY{l+s+sc}{\PYZsq{}y\PYZsq{}}\PY{o}{*}\PY{l+s+sc}{\PYZsq{}z\PYZsq{}}
\PY{c}{\PYZsh{} 4th mehod signature}
\PY{n}{f}\PY{p}{(}\PY{n}{x}\PY{o}{::}\PY{k+kt}{String}\PY{p}{)}\PY{+w}{ }\PY{o}{=}\PY{+w}{ }\PY{n}{x}\PY{o}{*}\PY{n}{x}
\end{Verbatim}
\end{tcolorbox}

            \begin{tcolorbox}[breakable, size=fbox, boxrule=.5pt, pad at break*=1mm, opacityfill=0]
\prompt{Out}{outcolor}{62}{\boxspacing}
\begin{Verbatim}[commandchars=\\\{\}]
f (generic function with 4 methods)
\end{Verbatim}
\end{tcolorbox}
        
    \begin{tcolorbox}[breakable, size=fbox, boxrule=1pt, pad at break*=1mm,colback=cellbackground, colframe=cellborder]
\prompt{In}{incolor}{63}{\boxspacing}
\begin{Verbatim}[commandchars=\\\{\}]
\PY{n}{methods}\PY{p}{(}\PY{n}{f}\PY{p}{)}
\end{Verbatim}
\end{tcolorbox}

            \begin{tcolorbox}[breakable, size=fbox, boxrule=.5pt, pad at break*=1mm, opacityfill=0]
\prompt{Out}{outcolor}{63}{\boxspacing}
\begin{Verbatim}[commandchars=\\\{\}]
\# 4 methods for generic function "f" from \textcolor{ansi-magenta}{Main}:
 [1] f(\textcolor{ansi-black-intense}{x}::\textbf{Int64})
\textcolor{ansi-black-intense}{     @} \textcolor{ansi-black-intense}{\underline{In[60]:2}}
 [2] f(\textcolor{ansi-black-intense}{x}::\textbf{Float64})
\textcolor{ansi-black-intense}{     @} \textcolor{ansi-black-intense}{\underline{In[61]:2}}
 [3] f(\textcolor{ansi-black-intense}{x}::\textbf{Char})
\textcolor{ansi-black-intense}{     @} \textcolor{ansi-black-intense}{\underline{In[62]:2}}
 [4] f(\textcolor{ansi-black-intense}{x}::\textbf{String})
\textcolor{ansi-black-intense}{     @} \textcolor{ansi-black-intense}{\underline{In[62]:4}}
\end{Verbatim}
\end{tcolorbox}
        
    \begin{tcolorbox}[breakable, size=fbox, boxrule=1pt, pad at break*=1mm,colback=cellbackground, colframe=cellborder]
\prompt{In}{incolor}{64}{\boxspacing}
\begin{Verbatim}[commandchars=\\\{\}]
\PY{n}{f}\PY{p}{(}\PY{l+m+mi}{1}\PY{p}{)}\PY{p}{,}\PY{+w}{ }\PY{n}{f}\PY{p}{(}\PY{l+m+mf}{1.}\PY{p}{)}\PY{p}{,}\PY{+w}{ }\PY{n}{f}\PY{p}{(}\PY{l+s+sc}{\PYZsq{}x\PYZsq{}}\PY{p}{)}\PY{p}{,}\PY{+w}{ }\PY{n}{f}\PY{p}{(}\PY{l+s}{\PYZdq{}}\PY{l+s}{abc}\PY{l+s}{\PYZdq{}}\PY{p}{)}
\end{Verbatim}
\end{tcolorbox}

            \begin{tcolorbox}[breakable, size=fbox, boxrule=.5pt, pad at break*=1mm, opacityfill=0]
\prompt{Out}{outcolor}{64}{\boxspacing}
\begin{Verbatim}[commandchars=\\\{\}]
(1, 2.0, "xyz", "abcabc")
\end{Verbatim}
\end{tcolorbox}
        
    \begin{tcolorbox}[breakable, size=fbox, boxrule=1pt, pad at break*=1mm,colback=cellbackground, colframe=cellborder]
\prompt{In}{incolor}{65}{\boxspacing}
\begin{Verbatim}[commandchars=\\\{\}]
\PY{n}{mycos}\PY{p}{(}\PY{n}{x}\PY{p}{)}\PY{+w}{ }\PY{o}{=}\PY{+w}{ }\PY{n}{cos}\PY{p}{(}\PY{n}{x}\PY{p}{)}
\PY{n}{mycos}\PY{p}{(}\PY{n}{adj}\PY{p}{,}\PY{+w}{ }\PY{n}{hyp}\PY{p}{)}\PY{+w}{ }\PY{o}{=}\PY{+w}{ }\PY{n}{adj}\PY{o}{/}\PY{n}{hyp}\PY{+w}{ }\PY{c}{\PYZsh{} Extension to `mycos` function}
\end{Verbatim}
\end{tcolorbox}

            \begin{tcolorbox}[breakable, size=fbox, boxrule=.5pt, pad at break*=1mm, opacityfill=0]
\prompt{Out}{outcolor}{65}{\boxspacing}
\begin{Verbatim}[commandchars=\\\{\}]
mycos (generic function with 2 methods)
\end{Verbatim}
\end{tcolorbox}
        
    \begin{tcolorbox}[breakable, size=fbox, boxrule=1pt, pad at break*=1mm,colback=cellbackground, colframe=cellborder]
\prompt{In}{incolor}{66}{\boxspacing}
\begin{Verbatim}[commandchars=\\\{\}]
\PY{n}{methods}\PY{p}{(}\PY{n}{mycos}\PY{p}{)}
\end{Verbatim}
\end{tcolorbox}

            \begin{tcolorbox}[breakable, size=fbox, boxrule=.5pt, pad at break*=1mm, opacityfill=0]
\prompt{Out}{outcolor}{66}{\boxspacing}
\begin{Verbatim}[commandchars=\\\{\}]
\# 2 methods for generic function "mycos" from \textcolor{ansi-magenta}{Main}:
 [1] mycos(\textcolor{ansi-black-intense}{x})
\textcolor{ansi-black-intense}{     @} \textcolor{ansi-black-intense}{\underline{In[65]:1}}
 [2] mycos(\textcolor{ansi-black-intense}{adj}, \textcolor{ansi-black-intense}{hyp})
\textcolor{ansi-black-intense}{     @} \textcolor{ansi-black-intense}{\underline{In[65]:2}}
\end{Verbatim}
\end{tcolorbox}
        
    \begin{tcolorbox}[breakable, size=fbox, boxrule=1pt, pad at break*=1mm,colback=cellbackground, colframe=cellborder]
\prompt{In}{incolor}{67}{\boxspacing}
\begin{Verbatim}[commandchars=\\\{\}]
\PY{n+nd}{@which}\PY{+w}{ }\PY{n}{mycos}\PY{p}{(}\PY{n+nb}{π}\PY{p}{)}
\end{Verbatim}
\end{tcolorbox}

            \begin{tcolorbox}[breakable, size=fbox, boxrule=.5pt, pad at break*=1mm, opacityfill=0]
\prompt{Out}{outcolor}{67}{\boxspacing}
\begin{Verbatim}[commandchars=\\\{\}]
mycos(\textcolor{ansi-black-intense}{x})
\textcolor{ansi-black-intense}{     @} \textcolor{ansi-black-intense}{Main} \textcolor{ansi-black-intense}{\underline{In[65]:1}}
\end{Verbatim}
\end{tcolorbox}
        
    \begin{tcolorbox}[breakable, size=fbox, boxrule=1pt, pad at break*=1mm,colback=cellbackground, colframe=cellborder]
\prompt{In}{incolor}{68}{\boxspacing}
\begin{Verbatim}[commandchars=\\\{\}]
\PY{n+nd}{@which}\PY{+w}{ }\PY{n}{mycos}\PY{p}{(}\PY{l+m+mi}{5}\PY{p}{,}\PY{+w}{ }\PY{l+m+mi}{3}\PY{p}{)}
\end{Verbatim}
\end{tcolorbox}

            \begin{tcolorbox}[breakable, size=fbox, boxrule=.5pt, pad at break*=1mm, opacityfill=0]
\prompt{Out}{outcolor}{68}{\boxspacing}
\begin{Verbatim}[commandchars=\\\{\}]
mycos(\textcolor{ansi-black-intense}{adj}, \textcolor{ansi-black-intense}{hyp})
\textcolor{ansi-black-intense}{     @} \textcolor{ansi-black-intense}{Main} \textcolor{ansi-black-intense}{\underline{In[65]:2}}
\end{Verbatim}
\end{tcolorbox}
        
    \begin{tcolorbox}[breakable, size=fbox, boxrule=1pt, pad at break*=1mm,colback=cellbackground, colframe=cellborder]
\prompt{In}{incolor}{69}{\boxspacing}
\begin{Verbatim}[commandchars=\\\{\}]
\PY{n}{mycos}\PY{p}{(}\PY{n}{adj}\PY{p}{,}\PY{+w}{ }\PY{n}{hyp}\PY{o}{=}\PY{l+m+mi}{10}\PY{p}{)}\PY{+w}{ }\PY{o}{=}\PY{+w}{ }\PY{n}{adj}\PY{o}{/}\PY{n}{hyp}
\end{Verbatim}
\end{tcolorbox}

            \begin{tcolorbox}[breakable, size=fbox, boxrule=.5pt, pad at break*=1mm, opacityfill=0]
\prompt{Out}{outcolor}{69}{\boxspacing}
\begin{Verbatim}[commandchars=\\\{\}]
mycos (generic function with 2 methods)
\end{Verbatim}
\end{tcolorbox}
        
    \begin{tcolorbox}[breakable, size=fbox, boxrule=1pt, pad at break*=1mm,colback=cellbackground, colframe=cellborder]
\prompt{In}{incolor}{70}{\boxspacing}
\begin{Verbatim}[commandchars=\\\{\}]
\PY{n+nd}{@which}\PY{+w}{ }\PY{n}{mycos}\PY{p}{(}\PY{n+nb}{π}\PY{p}{)}
\end{Verbatim}
\end{tcolorbox}

            \begin{tcolorbox}[breakable, size=fbox, boxrule=.5pt, pad at break*=1mm, opacityfill=0]
\prompt{Out}{outcolor}{70}{\boxspacing}
\begin{Verbatim}[commandchars=\\\{\}]
mycos(\textcolor{ansi-black-intense}{adj})
\textcolor{ansi-black-intense}{     @} \textcolor{ansi-black-intense}{Main} \textcolor{ansi-black-intense}{\underline{In[69]:1}}
\end{Verbatim}
\end{tcolorbox}
        
    \textbf{Function Chaining} applies a function to the preceding argument.

    \begin{tcolorbox}[breakable, size=fbox, boxrule=1pt, pad at break*=1mm,colback=cellbackground, colframe=cellborder]
\prompt{In}{incolor}{71}{\boxspacing}
\begin{Verbatim}[commandchars=\\\{\}]
\PY{n}{g}\PY{p}{(}\PY{n}{x}\PY{p}{)}\PY{+w}{ }\PY{o}{=}\PY{+w}{ }\PY{n}{x}\PY{o}{+}\PY{l+m+mi}{1}
\PY{n}{h}\PY{p}{(}\PY{n}{x}\PY{p}{)}\PY{+w}{ }\PY{o}{=}\PY{+w}{ }\PY{n}{x}\PY{o}{\PYZca{}}\PY{l+m+mi}{2}
\PY{n}{x}\PY{+w}{ }\PY{o}{=}\PY{+w}{ }\PY{l+m+mi}{2}\PY{+w}{ }\PY{o}{|\PYZgt{}}\PY{+w}{ }\PY{n}{g}\PY{+w}{ }\PY{o}{|\PYZgt{}}\PY{+w}{ }\PY{n}{h}
\end{Verbatim}
\end{tcolorbox}

            \begin{tcolorbox}[breakable, size=fbox, boxrule=.5pt, pad at break*=1mm, opacityfill=0]
\prompt{Out}{outcolor}{71}{\boxspacing}
\begin{Verbatim}[commandchars=\\\{\}]
9
\end{Verbatim}
\end{tcolorbox}
        
    \begin{tcolorbox}[breakable, size=fbox, boxrule=1pt, pad at break*=1mm,colback=cellbackground, colframe=cellborder]
\prompt{In}{incolor}{72}{\boxspacing}
\begin{Verbatim}[commandchars=\\\{\}]
\PY{l+s+sa}{md}\PY{l+s}{\PYZdq{}}\PY{l+s}{Another pssible way is t use `∘`\PYZus{}}\PY{l+s}{\PYZbs{}}\PY{l+s}{circ\PYZob{}tab\PYZcb{}\PYZus{} symbol}\PY{l+s}{\PYZdq{}}
\end{Verbatim}
\end{tcolorbox}
 
            
\prompt{Out}{outcolor}{72}{}
    
    Another pssible way is t use \texttt{∘}\emph{{\textbackslash}circ\{tab\}} symbol



    

    \begin{tcolorbox}[breakable, size=fbox, boxrule=1pt, pad at break*=1mm,colback=cellbackground, colframe=cellborder]
\prompt{In}{incolor}{73}{\boxspacing}
\begin{Verbatim}[commandchars=\\\{\}]
\PY{p}{(}\PY{n}{h∘g}\PY{p}{)}\PY{p}{(}\PY{l+m+mi}{2}\PY{p}{)}
\end{Verbatim}
\end{tcolorbox}

            \begin{tcolorbox}[breakable, size=fbox, boxrule=.5pt, pad at break*=1mm, opacityfill=0]
\prompt{Out}{outcolor}{73}{\boxspacing}
\begin{Verbatim}[commandchars=\\\{\}]
9
\end{Verbatim}
\end{tcolorbox}
        
    \begin{tcolorbox}[breakable, size=fbox, boxrule=1pt, pad at break*=1mm,colback=cellbackground, colframe=cellborder]
\prompt{In}{incolor}{74}{\boxspacing}
\begin{Verbatim}[commandchars=\\\{\}]
\PY{l+s+sa}{md}\PY{l+s}{\PYZdq{}}\PY{l+s}{Definition of a function can be done on the fly}\PY{l+s}{\PYZdq{}}
\end{Verbatim}
\end{tcolorbox}
 
            
\prompt{Out}{outcolor}{74}{}
    
    Definition of a function can be done on the fly



    

    \begin{tcolorbox}[breakable, size=fbox, boxrule=1pt, pad at break*=1mm,colback=cellbackground, colframe=cellborder]
\prompt{In}{incolor}{75}{\boxspacing}
\begin{Verbatim}[commandchars=\\\{\}]
\PY{n}{y}\PY{+w}{ }\PY{o}{=}\PY{+w}{ }\PY{l+m+mi}{5}\PY{+w}{ }\PY{o}{|\PYZgt{}}\PY{+w}{ }\PY{p}{(}\PY{n}{x}\PY{o}{\PYZhy{}\PYZgt{}}\PY{n}{x}\PY{o}{\PYZca{}}\PY{l+m+mi}{2}\PY{p}{)}\PY{+w}{ }\PY{o}{|\PYZgt{}}\PY{+w}{ }\PY{o}{√}
\end{Verbatim}
\end{tcolorbox}

            \begin{tcolorbox}[breakable, size=fbox, boxrule=.5pt, pad at break*=1mm, opacityfill=0]
\prompt{Out}{outcolor}{75}{\boxspacing}
\begin{Verbatim}[commandchars=\\\{\}]
5.0
\end{Verbatim}
\end{tcolorbox}
        
    \begin{tcolorbox}[breakable, size=fbox, boxrule=1pt, pad at break*=1mm,colback=cellbackground, colframe=cellborder]
\prompt{In}{incolor}{76}{\boxspacing}
\begin{Verbatim}[commandchars=\\\{\}]
\PY{l+s+sa}{md}\PY{l+s}{\PYZdq{}\PYZdq{}\PYZdq{}}
\PY{l+s}{**Metaprogramming:** Code is optimized by nature in }\PY{l+s}{\PYZdl{}}\PY{l+s}{\PYZob{}}\PY{l+s+se}{\PYZbs{}t}\PY{l+s}{t Julia\PYZcb{}}\PY{l+s}{\PYZdl{}}
\PY{l+s}{\PYZdq{}\PYZdq{}\PYZdq{}}
\end{Verbatim}
\end{tcolorbox}
 
            
\prompt{Out}{outcolor}{76}{}
    
    \textbf{Metaprogramming:} Code is optimized by nature in ${\tt Julia}$



    

    \begin{tcolorbox}[breakable, size=fbox, boxrule=1pt, pad at break*=1mm,colback=cellbackground, colframe=cellborder]
\prompt{In}{incolor}{77}{\boxspacing}
\begin{Verbatim}[commandchars=\\\{\}]
\PY{k}{function}\PY{+w}{ }\PY{n}{Foo}\PY{p}{(}\PY{n}{x}\PY{o}{::}\PY{k+kt}{Integer}\PY{p}{)}
\PY{+w}{    }\PY{n}{y}\PY{+w}{ }\PY{o}{=}\PY{+w}{ }\PY{n}{x}
\PY{+w}{    }\PY{k}{for}\PY{+w}{ }\PY{n}{i}\PY{o}{=}\PY{l+m+mi}{1}\PY{o}{:}\PY{l+m+mi}{100}
\PY{+w}{        }\PY{n}{y}\PY{+w}{ }\PY{o}{+=}\PY{+w}{ }\PY{n}{i}\PY{o}{\PYZca{}}\PY{l+m+mi}{2}
\PY{+w}{    }\PY{k}{end}
\PY{+w}{    }\PY{k}{return}\PY{+w}{ }\PY{n}{y}
\PY{k}{end}
\end{Verbatim}
\end{tcolorbox}

            \begin{tcolorbox}[breakable, size=fbox, boxrule=.5pt, pad at break*=1mm, opacityfill=0]
\prompt{Out}{outcolor}{77}{\boxspacing}
\begin{Verbatim}[commandchars=\\\{\}]
Foo (generic function with 1 method)
\end{Verbatim}
\end{tcolorbox}
        
    \begin{tcolorbox}[breakable, size=fbox, boxrule=1pt, pad at break*=1mm,colback=cellbackground, colframe=cellborder]
\prompt{In}{incolor}{78}{\boxspacing}
\begin{Verbatim}[commandchars=\\\{\}]
\PY{n+nd}{@code\PYZus{}llvm}\PY{+w}{ }\PY{n}{Foo}\PY{p}{(}\PY{l+m+mi}{3}\PY{p}{)}
\end{Verbatim}
\end{tcolorbox}

    \begin{Verbatim}[commandchars=\\\{\}]
\textcolor{ansi-black-intense}{;  @ In[77]:1 within `Foo`}
\textcolor{ansi-magenta-intense}{define} \textcolor{ansi-cyan}{i64} \textcolor{ansi-yellow-intense}{@julia\_Foo\_4232}\textcolor{ansi-yellow}{(}\textcolor{ansi-cyan}{i64}
\textcolor{ansi-magenta-intense}{signext} \%0\textcolor{ansi-yellow}{)} \#0 \textcolor{ansi-yellow}{\{}
\textcolor{ansi-red-intense}{top:}
\textcolor{ansi-black-intense}{;  @ In[77]:3 within `Foo`}
  \%1 = \textcolor{ansi-cyan-intense}{\textbf{add}} \textcolor{ansi-cyan}{i64} \%0, \textcolor{ansi-yellow}{338350}
\textcolor{ansi-black-intense}{;  @ In[77]:6 within `Foo`}
  \textcolor{ansi-cyan-intense}{\textbf{ret}} \textcolor{ansi-cyan}{i64} \%1
\textcolor{ansi-yellow}{\}}
    \end{Verbatim}

    \begin{tcolorbox}[breakable, size=fbox, boxrule=1pt, pad at break*=1mm,colback=cellbackground, colframe=cellborder]
\prompt{In}{incolor}{79}{\boxspacing}
\begin{Verbatim}[commandchars=\\\{\}]
\PY{o}{?}\PY{n+nd}{@code\PYZus{}llvm}
\end{Verbatim}
\end{tcolorbox}
 
            
\prompt{Out}{outcolor}{79}{}
    
    \begin{verbatim}
@code_llvm
\end{verbatim}
Evaluates the arguments to the function or macro call, determines their types, and calls \href{@ref}{\texttt{code\_llvm}} on the resulting expression. Set the optional keyword arguments \texttt{raw}, \texttt{dump\_module}, \texttt{debuginfo}, \texttt{optimize} by putting them and their value before the function call, like this:

\begin{verbatim}
@code_llvm raw=true dump_module=true debuginfo=:default f(x)
@code_llvm optimize=false f(x)
\end{verbatim}
\texttt{optimize} controls whether additional optimizations, such as inlining, are also applied. \texttt{raw} makes all metadata and dbg.* calls visible. \texttt{debuginfo} may be one of \texttt{:source} (default) or \texttt{:none},  to specify the verbosity of code comments. \texttt{dump\_module} prints the entire module that encapsulates the function.



    

    \hypertarget{plotting-data}{%
\subsubsection{Plotting Data}\label{plotting-data}}

Visualize variables using \({\tt Julia}\)'s plotting functions.

    \begin{tcolorbox}[breakable, size=fbox, boxrule=1pt, pad at break*=1mm,colback=cellbackground, colframe=cellborder]
\prompt{In}{incolor}{80}{\boxspacing}
\begin{Verbatim}[commandchars=\\\{\}]
\PY{p}{]}\PY{n}{add}\PY{+w}{ }\PY{n}{Plots}
\end{Verbatim}
\end{tcolorbox}

    \begin{Verbatim}[commandchars=\\\{\}]
\textcolor{ansi-green-intense}{\textbf{   Resolving}} package versions{\ldots}
\textcolor{ansi-green-intense}{\textbf{  No Changes}} to `\textasciitilde{}/MEGA/git-repos/infodev/Codes/Project.toml`
\textcolor{ansi-green-intense}{\textbf{  No Changes}} to `\textasciitilde{}/MEGA/git-
repos/infodev/Codes/Manifest.toml`
    \end{Verbatim}

    \begin{tcolorbox}[breakable, size=fbox, boxrule=1pt, pad at break*=1mm,colback=cellbackground, colframe=cellborder]
\prompt{In}{incolor}{81}{\boxspacing}
\begin{Verbatim}[commandchars=\\\{\}]
\PY{k}{using}\PY{+w}{ }\PY{n}{Plots}\PY{+w}{ }\PY{c}{\PYZsh{} GR is the default backend}
\end{Verbatim}
\end{tcolorbox}

    \begin{Verbatim}[commandchars=\\\{\}]
WARNING: using Plots.bar in module Main conflicts with an existing identifier.
    \end{Verbatim}

    \begin{tcolorbox}[breakable, size=fbox, boxrule=1pt, pad at break*=1mm,colback=cellbackground, colframe=cellborder]
\prompt{In}{incolor}{82}{\boxspacing}
\begin{Verbatim}[commandchars=\\\{\}]
\PY{n}{x}\PY{+w}{ }\PY{o}{=}\PY{+w}{ }\PY{l+m+mi}{1}\PY{o}{:}\PY{l+m+mf}{.1}\PY{o}{:}\PY{l+m+mi}{10}
\PY{n}{y}\PY{+w}{ }\PY{o}{=}\PY{+w}{ }\PY{n}{sin}\PY{o}{.}\PY{p}{(}\PY{n}{x}\PY{p}{)}
\PY{n}{z}\PY{+w}{ }\PY{o}{=}\PY{+w}{ }\PY{n}{cos}\PY{o}{.}\PY{p}{(}\PY{n}{x}\PY{p}{)}

\PY{n}{plot}\PY{p}{(}\PY{n}{x}\PY{p}{,}\PY{+w}{ }\PY{n}{y}\PY{p}{,}\PY{+w}{ }\PY{n}{label}\PY{o}{=}\PY{l+s}{\PYZdq{}}\PY{l+s}{sin(x)}\PY{l+s}{\PYZdq{}}\PY{p}{)}
\PY{n}{plot!}\PY{p}{(}\PY{n}{x}\PY{p}{,}\PY{+w}{ }\PY{n}{z}\PY{p}{,}\PY{+w}{ }\PY{n}{label}\PY{o}{=}\PY{l+s}{\PYZdq{}}\PY{l+s}{cos(x)}\PY{l+s}{\PYZdq{}}\PY{p}{)}\PY{+w}{ }\PY{c}{\PYZsh{} Hold on the previous plot}
\end{Verbatim}
\end{tcolorbox}
 
            
\prompt{Out}{outcolor}{82}{}
    
    \begin{center}
    \adjustimage{max size={0.9\linewidth}{0.9\paperheight}}{output_105_0.pdf}
    \end{center}
    { \hspace*{\fill} \\}
    

    \begin{tcolorbox}[breakable, size=fbox, boxrule=1pt, pad at break*=1mm,colback=cellbackground, colframe=cellborder]
\prompt{In}{incolor}{83}{\boxspacing}
\begin{Verbatim}[commandchars=\\\{\}]
\PY{l+s+sa}{md}\PY{l+s}{\PYZdq{}}\PY{l+s}{**Scatter Plot**}\PY{l+s}{\PYZdq{}}
\end{Verbatim}
\end{tcolorbox}
 
            
\prompt{Out}{outcolor}{83}{}
    
    \textbf{Scatter Plot}



    

    \begin{tcolorbox}[breakable, size=fbox, boxrule=1pt, pad at break*=1mm,colback=cellbackground, colframe=cellborder]
\prompt{In}{incolor}{84}{\boxspacing}
\begin{Verbatim}[commandchars=\\\{\}]
\PY{n}{x}\PY{+w}{ }\PY{o}{=}\PY{+w}{ }\PY{n}{range}\PY{p}{(}\PY{l+m+mi}{1}\PY{p}{,}\PY{+w}{ }\PY{l+m+mi}{10}\PY{p}{)}
\PY{n}{y}\PY{+w}{ }\PY{o}{=}\PY{+w}{ }\PY{n}{cos}\PY{o}{.}\PY{p}{(}\PY{n}{x}\PY{o}{.\PYZca{}}\PY{l+m+mi}{3}\PY{p}{)}
\PY{n}{scatter}\PY{p}{(}\PY{n}{x}\PY{p}{,}\PY{+w}{ }\PY{n}{y}\PY{p}{,}\PY{+w}{ }\PY{n}{legend}\PY{o}{=}\PY{n+nb}{false}\PY{p}{)}
\end{Verbatim}
\end{tcolorbox}
 
            
\prompt{Out}{outcolor}{84}{}
    
    \begin{center}
    \adjustimage{max size={0.9\linewidth}{0.9\paperheight}}{output_107_0.pdf}
    \end{center}
    { \hspace*{\fill} \\}
    

    \begin{tcolorbox}[breakable, size=fbox, boxrule=1pt, pad at break*=1mm,colback=cellbackground, colframe=cellborder]
\prompt{In}{incolor}{85}{\boxspacing}
\begin{Verbatim}[commandchars=\\\{\}]
\PY{l+s+sa}{md}\PY{l+s}{\PYZdq{}}\PY{l+s}{**Uniform Distribtion**}\PY{l+s}{\PYZdq{}}
\end{Verbatim}
\end{tcolorbox}
 
            
\prompt{Out}{outcolor}{85}{}
    
    \textbf{Uniform Distribtion}



    

    \begin{tcolorbox}[breakable, size=fbox, boxrule=1pt, pad at break*=1mm,colback=cellbackground, colframe=cellborder]
\prompt{In}{incolor}{86}{\boxspacing}
\begin{Verbatim}[commandchars=\\\{\}]
\PY{o}{?}\PY{n}{rand}\PY{p}{;}
\end{Verbatim}
\end{tcolorbox}

    \begin{tcolorbox}[breakable, size=fbox, boxrule=1pt, pad at break*=1mm,colback=cellbackground, colframe=cellborder]
\prompt{In}{incolor}{87}{\boxspacing}
\begin{Verbatim}[commandchars=\\\{\}]
\PY{n}{x}\PY{+w}{ }\PY{o}{=}\PY{+w}{ }\PY{n}{rand}\PY{p}{(}\PY{l+m+mi}{10}\PY{o}{\PYZca{}}\PY{l+m+mi}{5}\PY{p}{)}
\PY{n}{histogram}\PY{p}{(}\PY{n}{x}\PY{p}{,}\PY{+w}{ }\PY{n}{bins}\PY{o}{=}\PY{l+m+mi}{64}\PY{p}{,}\PY{+w}{ }\PY{n}{legend}\PY{o}{=}\PY{n+nb}{false}\PY{p}{)}
\end{Verbatim}
\end{tcolorbox}
 
            
\prompt{Out}{outcolor}{87}{}
    
    \begin{center}
    \adjustimage{max size={0.9\linewidth}{0.9\paperheight}}{output_110_0.pdf}
    \end{center}
    { \hspace*{\fill} \\}
    

    \begin{tcolorbox}[breakable, size=fbox, boxrule=1pt, pad at break*=1mm,colback=cellbackground, colframe=cellborder]
\prompt{In}{incolor}{88}{\boxspacing}
\begin{Verbatim}[commandchars=\\\{\}]
\PY{l+s+sa}{md}\PY{l+s}{\PYZdq{}}\PY{l+s}{**Normal Distribution**}\PY{l+s}{\PYZdq{}}
\end{Verbatim}
\end{tcolorbox}
 
            
\prompt{Out}{outcolor}{88}{}
    
    \textbf{Normal Distribution}



    

    \begin{tcolorbox}[breakable, size=fbox, boxrule=1pt, pad at break*=1mm,colback=cellbackground, colframe=cellborder]
\prompt{In}{incolor}{89}{\boxspacing}
\begin{Verbatim}[commandchars=\\\{\}]
\PY{o}{?}\PY{n}{randn}\PY{p}{;}
\end{Verbatim}
\end{tcolorbox}

    \begin{tcolorbox}[breakable, size=fbox, boxrule=1pt, pad at break*=1mm,colback=cellbackground, colframe=cellborder]
\prompt{In}{incolor}{90}{\boxspacing}
\begin{Verbatim}[commandchars=\\\{\}]
\PY{n}{x}\PY{+w}{ }\PY{o}{=}\PY{+w}{ }\PY{n}{randn}\PY{p}{(}\PY{l+m+mi}{10}\PY{o}{\PYZca{}}\PY{l+m+mi}{5}\PY{p}{)}
\PY{n}{histogram}\PY{p}{(}\PY{n}{x}\PY{p}{,}\PY{+w}{ }\PY{n}{bins}\PY{o}{=}\PY{l+m+mi}{64}\PY{p}{,}\PY{+w}{ }\PY{n}{legend}\PY{o}{=}\PY{n+nb}{false}\PY{p}{)}
\end{Verbatim}
\end{tcolorbox}
 
            
\prompt{Out}{outcolor}{90}{}
    
    \begin{center}
    \adjustimage{max size={0.9\linewidth}{0.9\paperheight}}{output_113_0.pdf}
    \end{center}
    { \hspace*{\fill} \\}
    

    \begin{tcolorbox}[breakable, size=fbox, boxrule=1pt, pad at break*=1mm,colback=cellbackground, colframe=cellborder]
\prompt{In}{incolor}{91}{\boxspacing}
\begin{Verbatim}[commandchars=\\\{\}]
\PY{l+s+sa}{md}\PY{l+s}{\PYZdq{}}\PY{l+s}{**Histogram in 2D**}\PY{l+s}{\PYZdq{}}
\end{Verbatim}
\end{tcolorbox}
 
            
\prompt{Out}{outcolor}{91}{}
    
    \textbf{Histogram in 2D}



    

    \begin{tcolorbox}[breakable, size=fbox, boxrule=1pt, pad at break*=1mm,colback=cellbackground, colframe=cellborder]
\prompt{In}{incolor}{92}{\boxspacing}
\begin{Verbatim}[commandchars=\\\{\}]
\PY{n}{x}\PY{+w}{ }\PY{o}{=}\PY{+w}{ }\PY{n}{randn}\PY{p}{(}\PY{l+m+mi}{10}\PY{o}{\PYZca{}}\PY{l+m+mi}{5}\PY{p}{)}
\PY{n}{y}\PY{+w}{ }\PY{o}{=}\PY{+w}{ }\PY{n}{randn}\PY{p}{(}\PY{l+m+mi}{10}\PY{o}{\PYZca{}}\PY{l+m+mi}{5}\PY{p}{)}
\PY{n}{histogram2d}\PY{p}{(}\PY{n}{x}\PY{p}{,}\PY{+w}{ }\PY{n}{y}\PY{p}{,}\PY{+w}{ }\PY{n}{bins}\PY{o}{=}\PY{p}{(}\PY{l+m+mi}{64}\PY{p}{,}\PY{+w}{ }\PY{l+m+mi}{64}\PY{p}{)}\PY{p}{)}
\end{Verbatim}
\end{tcolorbox}
 
            
\prompt{Out}{outcolor}{92}{}
    
    \begin{center}
    \adjustimage{max size={0.9\linewidth}{0.9\paperheight}}{output_115_0.pdf}
    \end{center}
    { \hspace*{\fill} \\}
    

    \hypertarget{importing-data}{%
\subsubsection{Importing Data}\label{importing-data}}

Bring data from external files into \({\tt Julia}\).

    Data is typically stored in files, such as \emph{CSV} or \emph{JSON}
files. In order to train and test machine learning models, the data
needs to be loaded into the program. Additionally, the results of the
training and testing process, such as model weights and performance
metrics, also need to be saved to files. Therefore, the ability to
manipulate files is essential for loading and saving data and model
information in the machine learning process.

    \begin{tcolorbox}[breakable, size=fbox, boxrule=1pt, pad at break*=1mm,colback=cellbackground, colframe=cellborder]
\prompt{In}{incolor}{93}{\boxspacing}
\begin{Verbatim}[commandchars=\\\{\}]
\PY{k}{using}\PY{+w}{ }\PY{n}{Pkg}
\PY{n}{Pkg}\PY{o}{.}\PY{n}{add}\PY{p}{(}\PY{l+s}{\PYZdq{}}\PY{l+s}{DataFrames}\PY{l+s}{\PYZdq{}}\PY{p}{)}
\PY{n}{Pkg}\PY{o}{.}\PY{n}{add}\PY{p}{(}\PY{l+s}{\PYZdq{}}\PY{l+s}{CSV}\PY{l+s}{\PYZdq{}}\PY{p}{)}
\end{Verbatim}
\end{tcolorbox}

    \begin{Verbatim}[commandchars=\\\{\}]
\textcolor{ansi-green-intense}{\textbf{   Resolving}} package versions{\ldots}
\textcolor{ansi-green-intense}{\textbf{  No Changes}} to `\textasciitilde{}/MEGA/git-repos/infodev/Codes/Project.toml`
\textcolor{ansi-green-intense}{\textbf{  No Changes}} to `\textasciitilde{}/MEGA/git-
repos/infodev/Codes/Manifest.toml`
\textcolor{ansi-green-intense}{\textbf{   Resolving}} package versions{\ldots}
\textcolor{ansi-green-intense}{\textbf{  No Changes}} to `\textasciitilde{}/MEGA/git-repos/infodev/Codes/Project.toml`
\textcolor{ansi-green-intense}{\textbf{  No Changes}} to `\textasciitilde{}/MEGA/git-
repos/infodev/Codes/Manifest.toml`
    \end{Verbatim}

    \begin{tcolorbox}[breakable, size=fbox, boxrule=1pt, pad at break*=1mm,colback=cellbackground, colframe=cellborder]
\prompt{In}{incolor}{94}{\boxspacing}
\begin{Verbatim}[commandchars=\\\{\}]
\PY{l+s+sa}{md}\PY{l+s}{\PYZdq{}}\PY{l+s}{Create new CSV file}\PY{l+s}{\PYZdq{}}
\end{Verbatim}
\end{tcolorbox}
 
            
\prompt{Out}{outcolor}{94}{}
    
    Create new CSV file



    

    \begin{tcolorbox}[breakable, size=fbox, boxrule=1pt, pad at break*=1mm,colback=cellbackground, colframe=cellborder]
\prompt{In}{incolor}{95}{\boxspacing}
\begin{Verbatim}[commandchars=\\\{\}]
\PY{k}{using}\PY{+w}{ }\PY{n}{CSV}\PY{p}{,}\PY{+w}{ }\PY{n}{DataFrames}
\end{Verbatim}
\end{tcolorbox}

    \begin{tcolorbox}[breakable, size=fbox, boxrule=1pt, pad at break*=1mm,colback=cellbackground, colframe=cellborder]
\prompt{In}{incolor}{96}{\boxspacing}
\begin{Verbatim}[commandchars=\\\{\}]
\PY{l+s+sa}{md}\PY{l+s}{\PYZdq{}}\PY{l+s}{`touch` command allows to create a file if it doesn\PYZsq{}t exist. Otherwise, it changes the file timestamps.}\PY{l+s}{\PYZdq{}}
\end{Verbatim}
\end{tcolorbox}
 
            
\prompt{Out}{outcolor}{96}{}
    
    \texttt{touch} command allows to create a file if it doesn't exist. Otherwise, it changes the file timestamps.



    

    \begin{tcolorbox}[breakable, size=fbox, boxrule=1pt, pad at break*=1mm,colback=cellbackground, colframe=cellborder]
\prompt{In}{incolor}{97}{\boxspacing}
\begin{Verbatim}[commandchars=\\\{\}]
\PY{n}{touch}\PY{p}{(}\PY{l+s}{\PYZdq{}}\PY{l+s}{test\PYZhy{}file.csv}\PY{l+s}{\PYZdq{}}\PY{p}{)}
\end{Verbatim}
\end{tcolorbox}

            \begin{tcolorbox}[breakable, size=fbox, boxrule=.5pt, pad at break*=1mm, opacityfill=0]
\prompt{Out}{outcolor}{97}{\boxspacing}
\begin{Verbatim}[commandchars=\\\{\}]
"test-file.csv"
\end{Verbatim}
\end{tcolorbox}
        
    \begin{tcolorbox}[breakable, size=fbox, boxrule=1pt, pad at break*=1mm,colback=cellbackground, colframe=cellborder]
\prompt{In}{incolor}{98}{\boxspacing}
\begin{Verbatim}[commandchars=\\\{\}]
\PY{p}{;}\PY{n}{ls}\PY{+w}{ }\PY{o}{\PYZhy{}}\PY{n}{la}\PY{+w}{ }\PY{n}{test}\PY{o}{\PYZhy{}}\PY{n}{file}\PY{o}{.}\PY{n}{csv}
\end{Verbatim}
\end{tcolorbox}

    \begin{Verbatim}[commandchars=\\\{\}]
-rw-rw-r-- 1 mhamdi mhamdi 66 Nov 29 21:36 test-file.csv
    \end{Verbatim}

    \begin{tcolorbox}[breakable, size=fbox, boxrule=1pt, pad at break*=1mm,colback=cellbackground, colframe=cellborder]
\prompt{In}{incolor}{99}{\boxspacing}
\begin{Verbatim}[commandchars=\\\{\}]
\PY{n}{file}\PY{+w}{ }\PY{o}{=}\PY{+w}{ }\PY{n}{open}\PY{p}{(}\PY{l+s}{\PYZdq{}}\PY{l+s}{test\PYZhy{}file.csv}\PY{l+s}{\PYZdq{}}\PY{p}{,}\PY{+w}{ }\PY{l+s}{\PYZdq{}}\PY{l+s}{w}\PY{l+s}{\PYZdq{}}\PY{p}{)}
\end{Verbatim}
\end{tcolorbox}

            \begin{tcolorbox}[breakable, size=fbox, boxrule=.5pt, pad at break*=1mm, opacityfill=0]
\prompt{Out}{outcolor}{99}{\boxspacing}
\begin{Verbatim}[commandchars=\\\{\}]
IOStream(<file test-file.csv>)
\end{Verbatim}
\end{tcolorbox}
        
    \begin{tcolorbox}[breakable, size=fbox, boxrule=1pt, pad at break*=1mm,colback=cellbackground, colframe=cellborder]
\prompt{In}{incolor}{100}{\boxspacing}
\begin{Verbatim}[commandchars=\\\{\}]
\PY{l+s+sa}{md}\PY{l+s}{\PYZdq{}}\PY{l+s}{Let\PYZsq{}s create some imaginary data}\PY{l+s}{\PYZdq{}}
\end{Verbatim}
\end{tcolorbox}
 
            
\prompt{Out}{outcolor}{100}{}
    
    Let's create some imaginary data



    

    \begin{tcolorbox}[breakable, size=fbox, boxrule=1pt, pad at break*=1mm,colback=cellbackground, colframe=cellborder]
\prompt{In}{incolor}{101}{\boxspacing}
\begin{Verbatim}[commandchars=\\\{\}]
\PY{n}{df}\PY{+w}{ }\PY{o}{=}\PY{+w}{ }\PY{n}{DataFrame}\PY{p}{(}
\PY{+w}{        }\PY{n}{Student}\PY{+w}{ }\PY{o}{=}\PY{+w}{ }\PY{p}{[}\PY{l+s}{\PYZdq{}}\PY{l+s}{Mohamed}\PY{l+s}{\PYZdq{}}\PY{p}{,}\PY{+w}{ }\PY{l+s}{\PYZdq{}}\PY{l+s}{Aymen}\PY{l+s}{\PYZdq{}}\PY{p}{,}\PY{+w}{ }\PY{l+s}{\PYZdq{}}\PY{l+s}{Rami}\PY{l+s}{\PYZdq{}}\PY{p}{,}\PY{+w}{ }\PY{l+s}{\PYZdq{}}\PY{l+s}{Ala}\PY{l+s}{\PYZdq{}}\PY{p}{]}\PY{p}{,}
\PY{+w}{        }\PY{n}{Id}\PY{+w}{ }\PY{o}{=}\PY{+w}{ }\PY{p}{[}\PY{l+m+mi}{1}\PY{p}{,}\PY{+w}{ }\PY{l+m+mi}{2}\PY{p}{,}\PY{+w}{ }\PY{l+m+mi}{3}\PY{p}{,}\PY{+w}{ }\PY{l+m+mi}{4}\PY{p}{]}\PY{p}{,}
\PY{+w}{        }\PY{n}{Marks}\PY{+w}{ }\PY{o}{=}\PY{+w}{ }\PY{p}{[}\PY{l+m+mi}{18}\PY{p}{,}\PY{+w}{ }\PY{l+m+mi}{7}\PY{p}{,}\PY{+w}{ }\PY{l+m+mi}{12}\PY{p}{,}\PY{+w}{ }\PY{l+m+mf}{5.5}\PY{p}{]}
\PY{+w}{        }\PY{p}{)}
\end{Verbatim}
\end{tcolorbox}
 
            
\prompt{Out}{outcolor}{101}{}
    
    \begin{tabular}{r|ccc}
	& Student & Id & Marks\\
	\hline
	& String & Int64 & Float64\\
	\hline
	1 & Mohamed & 1 & 18.0 \\
	2 & Aymen & 2 & 7.0 \\
	3 & Rami & 3 & 12.0 \\
	4 & Ala & 4 & 5.5 \\
\end{tabular}


    

    \begin{tcolorbox}[breakable, size=fbox, boxrule=1pt, pad at break*=1mm,colback=cellbackground, colframe=cellborder]
\prompt{In}{incolor}{102}{\boxspacing}
\begin{Verbatim}[commandchars=\\\{\}]
\PY{l+s+sa}{md}\PY{l+s}{\PYZdq{}}\PY{l+s}{Write `df` to file}\PY{l+s}{\PYZdq{}}
\end{Verbatim}
\end{tcolorbox}
 
            
\prompt{Out}{outcolor}{102}{}
    
    Write \texttt{df} to file



    

    \begin{tcolorbox}[breakable, size=fbox, boxrule=1pt, pad at break*=1mm,colback=cellbackground, colframe=cellborder]
\prompt{In}{incolor}{103}{\boxspacing}
\begin{Verbatim}[commandchars=\\\{\}]
\PY{n}{CSV}\PY{o}{.}\PY{n}{write}\PY{p}{(}\PY{l+s}{\PYZdq{}}\PY{l+s}{test\PYZhy{}file.csv}\PY{l+s}{\PYZdq{}}\PY{p}{,}\PY{+w}{ }\PY{n}{df}\PY{p}{)}
\end{Verbatim}
\end{tcolorbox}

            \begin{tcolorbox}[breakable, size=fbox, boxrule=.5pt, pad at break*=1mm, opacityfill=0]
\prompt{Out}{outcolor}{103}{\boxspacing}
\begin{Verbatim}[commandchars=\\\{\}]
"test-file.csv"
\end{Verbatim}
\end{tcolorbox}
        
    \begin{tcolorbox}[breakable, size=fbox, boxrule=1pt, pad at break*=1mm,colback=cellbackground, colframe=cellborder]
\prompt{In}{incolor}{104}{\boxspacing}
\begin{Verbatim}[commandchars=\\\{\}]
\PY{l+s+sa}{md}\PY{l+s}{\PYZdq{}}\PY{l+s}{Open the CSV file and add some contents. See what happens when we load it again.}\PY{l+s}{\PYZdq{}}
\end{Verbatim}
\end{tcolorbox}
 
            
\prompt{Out}{outcolor}{104}{}
    
    Open the CSV file and add some contents. See what happens when we load it again.



    

    \begin{tcolorbox}[breakable, size=fbox, boxrule=1pt, pad at break*=1mm,colback=cellbackground, colframe=cellborder]
\prompt{In}{incolor}{105}{\boxspacing}
\begin{Verbatim}[commandchars=\\\{\}]
\PY{n}{CSV}\PY{o}{.}\PY{n}{read}\PY{p}{(}\PY{l+s}{\PYZdq{}}\PY{l+s}{test\PYZhy{}file.csv}\PY{l+s}{\PYZdq{}}\PY{p}{,}\PY{+w}{ }\PY{n}{DataFrame}\PY{p}{)}
\end{Verbatim}
\end{tcolorbox}
 
            
\prompt{Out}{outcolor}{105}{}
    
    \begin{tabular}{r|ccc}
	& Student & Id & Marks\\
	\hline
	& String7 & Int64 & Float64\\
	\hline
	1 & Mohamed & 1 & 18.0 \\
	2 & Aymen & 2 & 7.0 \\
	3 & Rami & 3 & 12.0 \\
	4 & Ala & 4 & 5.5 \\
\end{tabular}


    

    \hypertarget{logical-arrays}{%
\subsubsection{Logical Arrays}\label{logical-arrays}}

Use logical expressions to help extracting elements of interest from
\({\tt Julia}\) arrays.

    \begin{tcolorbox}[breakable, size=fbox, boxrule=1pt, pad at break*=1mm,colback=cellbackground, colframe=cellborder]
\prompt{In}{incolor}{106}{\boxspacing}
\begin{Verbatim}[commandchars=\\\{\}]
\PY{n}{x}\PY{+w}{ }\PY{o}{=}\PY{+w}{ }\PY{p}{[}\PY{l+m+mi}{1}\PY{p}{,}\PY{+w}{ }\PY{l+m+mi}{2}\PY{p}{,}\PY{+w}{ }\PY{o}{\PYZhy{}}\PY{l+m+mi}{5}\PY{p}{,}\PY{+w}{ }\PY{l+m+mf}{7.2}\PY{p}{,}\PY{+w}{ }\PY{l+m+mi}{3}\PY{n+nb}{im}\PY{p}{]}
\PY{n}{println}\PY{p}{(}\PY{n}{x}\PY{p}{)}
\PY{n}{typeof}\PY{p}{(}\PY{n}{x}\PY{p}{)}
\end{Verbatim}
\end{tcolorbox}

    \begin{Verbatim}[commandchars=\\\{\}]
ComplexF64[1.0 + 0.0im, 2.0 + 0.0im, -5.0 + 0.0im, 7.2 + 0.0im, 0.0 + 3.0im]
    \end{Verbatim}

            \begin{tcolorbox}[breakable, size=fbox, boxrule=.5pt, pad at break*=1mm, opacityfill=0]
\prompt{Out}{outcolor}{106}{\boxspacing}
\begin{Verbatim}[commandchars=\\\{\}]
Vector\{ComplexF64\}\textcolor{ansi-black-intense}{ (alias for }\textcolor{ansi-black-intense}{Array\{Complex\{Float64\},
1\}}\textcolor{ansi-black-intense}{)}
\end{Verbatim}
\end{tcolorbox}
        
    \begin{tcolorbox}[breakable, size=fbox, boxrule=1pt, pad at break*=1mm,colback=cellbackground, colframe=cellborder]
\prompt{In}{incolor}{107}{\boxspacing}
\begin{Verbatim}[commandchars=\\\{\}]
\PY{n}{idx}\PY{+w}{ }\PY{o}{=}\PY{+w}{ }\PY{p}{[}\PY{n+nb}{false}\PY{p}{,}\PY{+w}{ }\PY{n+nb}{true}\PY{p}{,}\PY{+w}{ }\PY{n+nb}{false}\PY{p}{,}\PY{+w}{ }\PY{n+nb}{false}\PY{p}{,}\PY{+w}{ }\PY{n+nb}{true}\PY{p}{]}
\PY{n}{print}\PY{p}{(}\PY{n}{x}\PY{p}{[}\PY{n}{idx}\PY{p}{]}\PY{p}{)}
\end{Verbatim}
\end{tcolorbox}

    \begin{Verbatim}[commandchars=\\\{\}]
ComplexF64[2.0 + 0.0im, 0.0 + 3.0im]
    \end{Verbatim}

    \begin{tcolorbox}[breakable, size=fbox, boxrule=1pt, pad at break*=1mm,colback=cellbackground, colframe=cellborder]
\prompt{In}{incolor}{108}{\boxspacing}
\begin{Verbatim}[commandchars=\\\{\}]
\PY{n}{M}\PY{+w}{ }\PY{o}{=}\PY{+w}{ }\PY{k+kt}{Array}\PY{p}{\PYZob{}}\PY{k+kt}{Float64}\PY{p}{,}\PY{+w}{ }\PY{l+m+mi}{2}\PY{p}{\PYZcb{}}\PY{p}{(}\PY{n+nb}{undef}\PY{p}{,}\PY{+w}{ }\PY{l+m+mi}{5}\PY{p}{,}\PY{+w}{ }\PY{l+m+mi}{4}\PY{p}{)}
\end{Verbatim}
\end{tcolorbox}

            \begin{tcolorbox}[breakable, size=fbox, boxrule=.5pt, pad at break*=1mm, opacityfill=0]
\prompt{Out}{outcolor}{108}{\boxspacing}
\begin{Verbatim}[commandchars=\\\{\}]
5×4 Matrix\{Float64\}:
 6.90677e-310  6.90677e-310  6.90677e-310  6.90676e-310
 6.90677e-310  6.90677e-310  6.90677e-310  6.90676e-310
 6.90677e-310  6.90677e-310  6.90676e-310  6.90676e-310
 6.90677e-310  6.90677e-310  6.90676e-310  6.90676e-310
 6.90677e-310  6.90677e-310  6.90677e-310  6.90677e-310
\end{Verbatim}
\end{tcolorbox}
        
    \begin{tcolorbox}[breakable, size=fbox, boxrule=1pt, pad at break*=1mm,colback=cellbackground, colframe=cellborder]
\prompt{In}{incolor}{109}{\boxspacing}
\begin{Verbatim}[commandchars=\\\{\}]
\PY{n}{row\PYZus{}idx}\PY{+w}{ }\PY{o}{=}\PY{+w}{ }\PY{p}{[}\PY{n+nb}{true}\PY{p}{,}\PY{+w}{ }\PY{n+nb}{false}\PY{p}{,}\PY{+w}{ }\PY{n+nb}{true}\PY{p}{,}\PY{+w}{ }\PY{n+nb}{true}\PY{p}{,}\PY{+w}{ }\PY{n+nb}{false}\PY{p}{]}\PY{p}{;}
\PY{n}{col\PYZus{}idx}\PY{+w}{ }\PY{o}{=}\PY{+w}{ }\PY{p}{[}\PY{n+nb}{false}\PY{p}{,}\PY{+w}{ }\PY{n+nb}{true}\PY{p}{,}\PY{+w}{ }\PY{n+nb}{true}\PY{p}{,}\PY{+w}{ }\PY{n+nb}{false}\PY{p}{]}\PY{p}{;}
\end{Verbatim}
\end{tcolorbox}

    \begin{tcolorbox}[breakable, size=fbox, boxrule=1pt, pad at break*=1mm,colback=cellbackground, colframe=cellborder]
\prompt{In}{incolor}{110}{\boxspacing}
\begin{Verbatim}[commandchars=\\\{\}]
\PY{n}{M}\PY{p}{[}\PY{n}{row\PYZus{}idx}\PY{p}{,}\PY{+w}{ }\PY{o}{:}\PY{p}{]}
\end{Verbatim}
\end{tcolorbox}

            \begin{tcolorbox}[breakable, size=fbox, boxrule=.5pt, pad at break*=1mm, opacityfill=0]
\prompt{Out}{outcolor}{110}{\boxspacing}
\begin{Verbatim}[commandchars=\\\{\}]
3×4 Matrix\{Float64\}:
 6.90677e-310  6.90677e-310  6.90677e-310  6.90676e-310
 6.90677e-310  6.90677e-310  6.90676e-310  6.90676e-310
 6.90677e-310  6.90677e-310  6.90676e-310  6.90676e-310
\end{Verbatim}
\end{tcolorbox}
        
    \begin{tcolorbox}[breakable, size=fbox, boxrule=1pt, pad at break*=1mm,colback=cellbackground, colframe=cellborder]
\prompt{In}{incolor}{111}{\boxspacing}
\begin{Verbatim}[commandchars=\\\{\}]
\PY{n}{M}\PY{p}{[}\PY{o}{:}\PY{p}{,}\PY{+w}{ }\PY{n}{col\PYZus{}idx}\PY{p}{]}
\end{Verbatim}
\end{tcolorbox}

            \begin{tcolorbox}[breakable, size=fbox, boxrule=.5pt, pad at break*=1mm, opacityfill=0]
\prompt{Out}{outcolor}{111}{\boxspacing}
\begin{Verbatim}[commandchars=\\\{\}]
5×2 Matrix\{Float64\}:
 6.90677e-310  6.90677e-310
 6.90677e-310  6.90677e-310
 6.90677e-310  6.90676e-310
 6.90677e-310  6.90676e-310
 6.90677e-310  6.90677e-310
\end{Verbatim}
\end{tcolorbox}
        
    \begin{tcolorbox}[breakable, size=fbox, boxrule=1pt, pad at break*=1mm,colback=cellbackground, colframe=cellborder]
\prompt{In}{incolor}{112}{\boxspacing}
\begin{Verbatim}[commandchars=\\\{\}]
\PY{n}{M}\PY{p}{[}\PY{n}{row\PYZus{}idx}\PY{p}{,}\PY{+w}{ }\PY{n}{col\PYZus{}idx}\PY{p}{]}
\end{Verbatim}
\end{tcolorbox}

            \begin{tcolorbox}[breakable, size=fbox, boxrule=.5pt, pad at break*=1mm, opacityfill=0]
\prompt{Out}{outcolor}{112}{\boxspacing}
\begin{Verbatim}[commandchars=\\\{\}]
3×2 Matrix\{Float64\}:
 6.90677e-310  6.90677e-310
 6.90677e-310  6.90676e-310
 6.90677e-310  6.90676e-310
\end{Verbatim}
\end{tcolorbox}
        
    \hypertarget{programming}{%
\subsubsection{Programming}\label{programming}}

Write programs that execute code based on some condition.

    \begin{tcolorbox}[breakable, size=fbox, boxrule=1pt, pad at break*=1mm,colback=cellbackground, colframe=cellborder]
\prompt{In}{incolor}{113}{\boxspacing}
\begin{Verbatim}[commandchars=\\\{\}]
\PY{l+s+sa}{md}\PY{l+s}{\PYZdq{}}\PY{l+s}{**Conditional Evaluation**}\PY{l+s}{\PYZdq{}}
\end{Verbatim}
\end{tcolorbox}
 
            
\prompt{Out}{outcolor}{113}{}
    
    \textbf{Conditional Evaluation}



    

    \begin{tcolorbox}[breakable, size=fbox, boxrule=1pt, pad at break*=1mm,colback=cellbackground, colframe=cellborder]
\prompt{In}{incolor}{114}{\boxspacing}
\begin{Verbatim}[commandchars=\\\{\}]
\PY{n}{a}\PY{p}{,}\PY{+w}{ }\PY{n}{b}\PY{+w}{ }\PY{o}{=}\PY{+w}{ }\PY{n+nb}{π}\PY{p}{,}\PY{+w}{ }\PY{n+nb}{π}
\PY{k}{if}\PY{+w}{ }\PY{n}{a}\PY{+w}{ }\PY{o}{\PYZlt{}}\PY{+w}{ }\PY{n}{b}
\PY{+w}{    }\PY{n}{println}\PY{p}{(}\PY{l+s}{\PYZdq{}}\PY{l+s+si}{\PYZdl{}a}\PY{l+s}{ is less than }\PY{l+s+si}{\PYZdl{}b}\PY{l+s}{\PYZdq{}}\PY{p}{)}
\PY{k}{elseif}\PY{+w}{ }\PY{n}{a}\PY{+w}{ }\PY{o}{\PYZgt{}}\PY{+w}{ }\PY{n}{b}
\PY{+w}{    }\PY{n}{println}\PY{p}{(}\PY{l+s}{\PYZdq{}}\PY{l+s+si}{\PYZdl{}a}\PY{l+s}{ is greater than }\PY{l+s+si}{\PYZdl{}b}\PY{l+s}{\PYZdq{}}\PY{p}{)}
\PY{k}{else}
\PY{+w}{    }\PY{n}{println}\PY{p}{(}\PY{l+s}{\PYZdq{}}\PY{l+s+si}{\PYZdl{}a}\PY{l+s}{ is equal to }\PY{l+s+si}{\PYZdl{}b}\PY{l+s}{\PYZdq{}}\PY{p}{)}
\PY{k}{end}
\end{Verbatim}
\end{tcolorbox}

    \begin{Verbatim}[commandchars=\\\{\}]
π is equal to π
    \end{Verbatim}

    \begin{tcolorbox}[breakable, size=fbox, boxrule=1pt, pad at break*=1mm,colback=cellbackground, colframe=cellborder]
\prompt{In}{incolor}{115}{\boxspacing}
\begin{Verbatim}[commandchars=\\\{\}]
\PY{l+s+sa}{md}\PY{l+s}{\PYZdq{}}\PY{l+s}{**`While` Loop**}\PY{l+s}{\PYZdq{}}
\end{Verbatim}
\end{tcolorbox}
 
            
\prompt{Out}{outcolor}{115}{}
    
    \textbf{\texttt{While} Loop}



    

    \begin{tcolorbox}[breakable, size=fbox, boxrule=1pt, pad at break*=1mm,colback=cellbackground, colframe=cellborder]
\prompt{In}{incolor}{116}{\boxspacing}
\begin{Verbatim}[commandchars=\\\{\}]
\PY{n}{fruits}\PY{+w}{ }\PY{o}{=}\PY{+w}{ }\PY{p}{[}\PY{l+s}{\PYZdq{}}\PY{l+s}{Blueberry}\PY{l+s}{\PYZdq{}}\PY{p}{,}\PY{+w}{ }\PY{l+s}{\PYZdq{}}\PY{l+s}{Orange}\PY{l+s}{\PYZdq{}}\PY{p}{,}\PY{+w}{ }\PY{l+s}{\PYZdq{}}\PY{l+s}{Banana}\PY{l+s}{\PYZdq{}}\PY{p}{,}\PY{+w}{ }\PY{l+s}{\PYZdq{}}\PY{l+s}{Raspberry}\PY{l+s}{\PYZdq{}}\PY{p}{,}\PY{+w}{ }\PY{l+s}{\PYZdq{}}\PY{l+s}{Strawberry}\PY{l+s}{\PYZdq{}}\PY{p}{]}
\PY{n}{iter}\PY{+w}{ }\PY{o}{=}\PY{+w}{ }\PY{l+m+mi}{1}
\PY{k}{while}\PY{+w}{ }\PY{n}{iter}\PY{+w}{ }\PY{o}{≤}\PY{+w}{ }\PY{n}{length}\PY{p}{(}\PY{n}{fruits}\PY{p}{)}
\PY{+w}{    }\PY{n}{println}\PY{p}{(}\PY{l+s}{\PYZdq{}}\PY{l+s}{Item \PYZsh{}}\PY{l+s+si}{\PYZdl{}iter}\PY{l+s}{ is }\PY{l+s+si}{\PYZdl{}}\PY{p}{(}\PY{n}{fruits}\PY{p}{[}\PY{n}{iter}\PY{p}{]}\PY{p}{)}\PY{l+s}{\PYZdq{}}\PY{p}{)}
\PY{+w}{    }\PY{n}{iter}\PY{+w}{ }\PY{o}{+=}\PY{l+m+mi}{1}
\PY{k}{end}
\end{Verbatim}
\end{tcolorbox}

    \begin{Verbatim}[commandchars=\\\{\}]
Item \#1 is Blueberry
Item \#2 is Orange
Item \#3 is Banana
Item \#4 is Raspberry
Item \#5 is Strawberry
    \end{Verbatim}

    \begin{tcolorbox}[breakable, size=fbox, boxrule=1pt, pad at break*=1mm,colback=cellbackground, colframe=cellborder]
\prompt{In}{incolor}{117}{\boxspacing}
\begin{Verbatim}[commandchars=\\\{\}]
\PY{l+s+sa}{md}\PY{l+s}{\PYZdq{}}\PY{l+s}{**`For` Loop**}\PY{l+s}{\PYZdq{}}
\end{Verbatim}
\end{tcolorbox}
 
            
\prompt{Out}{outcolor}{117}{}
    
    \textbf{\texttt{For} Loop}



    

    \begin{tcolorbox}[breakable, size=fbox, boxrule=1pt, pad at break*=1mm,colback=cellbackground, colframe=cellborder]
\prompt{In}{incolor}{118}{\boxspacing}
\begin{Verbatim}[commandchars=\\\{\}]
\PY{n}{vegetables}\PY{+w}{ }\PY{o}{=}\PY{+w}{ }\PY{p}{[}\PY{l+s}{\PYZdq{}}\PY{l+s}{Broccoli}\PY{l+s}{\PYZdq{}}\PY{p}{,}\PY{+w}{ }\PY{l+s}{\PYZdq{}}\PY{l+s}{Garlic}\PY{l+s}{\PYZdq{}}\PY{p}{,}\PY{+w}{ }\PY{l+s}{\PYZdq{}}\PY{l+s}{Mushrooms}\PY{l+s}{\PYZdq{}}\PY{p}{,}\PY{+w}{ }\PY{l+s}{\PYZdq{}}\PY{l+s}{Potatoes}\PY{l+s}{\PYZdq{}}\PY{p}{,}\PY{+w}{ }\PY{l+s}{\PYZdq{}}\PY{l+s}{Tomatoes}\PY{l+s}{\PYZdq{}}\PY{p}{]}
\PY{n}{i}\PY{+w}{ }\PY{o}{=}\PY{+w}{ }\PY{l+m+mi}{1}
\PY{k}{for}\PY{+w}{ }\PY{n}{item}\PY{+w}{ }\PY{k}{in}\PY{+w}{ }\PY{n}{vegetables}
\PY{+w}{    }\PY{n}{println}\PY{p}{(}\PY{l+s}{\PYZdq{}}\PY{l+s}{Item \PYZsh{}}\PY{l+s+si}{\PYZdl{}i}\PY{l+s}{ is }\PY{l+s+si}{\PYZdl{}item}\PY{l+s}{\PYZdq{}}\PY{p}{)}
\PY{+w}{    }\PY{n}{i}\PY{+w}{ }\PY{o}{+=}\PY{+w}{ }\PY{l+m+mi}{1}
\PY{k}{end}
\end{Verbatim}
\end{tcolorbox}

    \begin{Verbatim}[commandchars=\\\{\}]
Item \#1 is Broccoli
Item \#2 is Garlic
Item \#3 is Mushrooms
Item \#4 is Potatoes
Item \#5 is Tomatoes
    \end{Verbatim}

    \hypertarget{final-project}{%
\subsubsection{Final Project}\label{final-project}}

Bring together concepts that you have learned with a project.

    This simple project consists of implementing a basic calculator. This
latter could have the ability to perform basic arithmetic operations
like \emph{addition}, \emph{subtraction}, \emph{multiplication}, and
\emph{division}.

Here are the steps to be followed: 1. Create a function called
\texttt{calculator()} that takes two arguments, \texttt{x} and
\texttt{y}, and a char operation that specifies which operation to
perform. 1. Use an \texttt{if-else} statement to check the value of
operation. Depending on the value of operation, call the appropriate
function to perform the calculation. 1. Test the calculator function by
calling it with different values for \texttt{x}, \texttt{y}, and
operation and printing the result. 1. Once the basic calculator is
working, we can improve it by adding more functionality such as handling
decimals and negative numbers, or implementing more advanced operations
such as square root, power, trigonometry and so on. 1. Finally, we could
also experiment with different input types, such as command line
arguments or a graphical user interface.

    \begin{tcolorbox}[breakable, size=fbox, boxrule=1pt, pad at break*=1mm,colback=cellbackground, colframe=cellborder]
\prompt{In}{incolor}{119}{\boxspacing}
\begin{Verbatim}[commandchars=\\\{\}]
\PY{l+s+sa}{md}\PY{l+s}{\PYZdq{}}\PY{l+s}{Here is an example of how the basic calculator function could look like:}\PY{l+s}{\PYZdq{}}
\end{Verbatim}
\end{tcolorbox}
 
            
\prompt{Out}{outcolor}{119}{}
    
    Here is an example of how the basic calculator function could look like:



    

    \begin{tcolorbox}[breakable, size=fbox, boxrule=1pt, pad at break*=1mm,colback=cellbackground, colframe=cellborder]
\prompt{In}{incolor}{120}{\boxspacing}
\begin{Verbatim}[commandchars=\\\{\}]
\PY{k}{function}\PY{+w}{ }\PY{n}{calculator}\PY{p}{(}\PY{n}{x}\PY{o}{::}\PY{k+kt}{Number}\PY{p}{,}\PY{+w}{ }\PY{n}{y}\PY{o}{::}\PY{k+kt}{Number}\PY{p}{,}\PY{+w}{ }\PY{n}{op}\PY{o}{::}\PY{k+kt}{Char}\PY{p}{)}
\PY{+w}{    }\PY{k}{if}\PY{+w}{ }\PY{n}{op}\PY{+w}{ }\PY{o}{==}\PY{+w}{ }\PY{l+s+sc}{\PYZsq{}+\PYZsq{}}
\PY{+w}{        }\PY{k}{return}\PY{+w}{ }\PY{n}{x}\PY{+w}{ }\PY{o}{+}\PY{+w}{ }\PY{n}{y}
\PY{+w}{    }\PY{k}{elseif}\PY{+w}{ }\PY{n}{op}\PY{+w}{ }\PY{o}{==}\PY{+w}{ }\PY{l+s+sc}{\PYZsq{}\PYZhy{}\PYZsq{}}
\PY{+w}{        }\PY{k}{return}\PY{+w}{ }\PY{n}{x}\PY{+w}{ }\PY{o}{\PYZhy{}}\PY{+w}{ }\PY{n}{y}
\PY{+w}{    }\PY{k}{elseif}\PY{+w}{ }\PY{n}{op}\PY{+w}{ }\PY{o}{==}\PY{+w}{ }\PY{l+s+sc}{\PYZsq{}*\PYZsq{}}
\PY{+w}{        }\PY{k}{return}\PY{+w}{ }\PY{n}{x}\PY{+w}{ }\PY{o}{*}\PY{+w}{ }\PY{n}{y}
\PY{+w}{    }\PY{k}{elseif}\PY{+w}{ }\PY{n}{op}\PY{+w}{ }\PY{k}{in}\PY{+w}{ }\PY{p}{[}\PY{l+s+sc}{\PYZsq{}/\PYZsq{}}\PY{p}{,}\PY{+w}{ }\PY{l+s+sc}{\PYZsq{}÷\PYZsq{}}\PY{p}{]}
\PY{+w}{        }\PY{k}{return}\PY{+w}{ }\PY{n}{x}\PY{+w}{ }\PY{o}{/}\PY{+w}{ }\PY{n}{y}
\PY{+w}{    }\PY{k}{else}
\PY{+w}{        }\PY{k}{return}\PY{+w}{ }\PY{l+s}{\PYZdq{}}\PY{l+s}{INVALID OPERATION}\PY{l+s}{\PYZdq{}}
\PY{+w}{    }\PY{k}{end}
\PY{k}{end}
\end{Verbatim}
\end{tcolorbox}

            \begin{tcolorbox}[breakable, size=fbox, boxrule=.5pt, pad at break*=1mm, opacityfill=0]
\prompt{Out}{outcolor}{120}{\boxspacing}
\begin{Verbatim}[commandchars=\\\{\}]
calculator (generic function with 1 method)
\end{Verbatim}
\end{tcolorbox}
        
    \begin{tcolorbox}[breakable, size=fbox, boxrule=1pt, pad at break*=1mm,colback=cellbackground, colframe=cellborder]
\prompt{In}{incolor}{121}{\boxspacing}
\begin{Verbatim}[commandchars=\\\{\}]
\PY{n}{println}\PY{p}{(}\PY{l+s}{\PYZdq{}}\PY{l+s}{Summation is }\PY{l+s+si}{\PYZdl{}}\PY{p}{(}\PY{n}{calculator}\PY{p}{(}\PY{l+m+mi}{5}\PY{p}{,}\PY{+w}{ }\PY{l+m+mi}{3}\PY{p}{,}\PY{+w}{ }\PY{l+s+sc}{\PYZsq{}+\PYZsq{}}\PY{p}{)}\PY{p}{)}\PY{l+s}{\PYZdq{}}\PY{p}{)}
\PY{n}{println}\PY{p}{(}\PY{l+s}{\PYZdq{}}\PY{l+s}{Subtraction is }\PY{l+s+si}{\PYZdl{}}\PY{p}{(}\PY{n}{calculator}\PY{p}{(}\PY{l+m+mi}{5}\PY{p}{,}\PY{+w}{ }\PY{l+m+mi}{3}\PY{p}{,}\PY{+w}{ }\PY{l+s+sc}{\PYZsq{}\PYZhy{}\PYZsq{}}\PY{p}{)}\PY{p}{)}\PY{l+s}{\PYZdq{}}\PY{p}{)}
\PY{n}{println}\PY{p}{(}\PY{l+s}{\PYZdq{}}\PY{l+s}{Multiplication is }\PY{l+s+si}{\PYZdl{}}\PY{p}{(}\PY{n}{calculator}\PY{p}{(}\PY{l+m+mi}{5}\PY{p}{,}\PY{+w}{ }\PY{l+m+mi}{3}\PY{p}{,}\PY{+w}{ }\PY{l+s+sc}{\PYZsq{}*\PYZsq{}}\PY{p}{)}\PY{p}{)}\PY{l+s}{\PYZdq{}}\PY{p}{)}
\PY{n}{println}\PY{p}{(}\PY{l+s}{\PYZdq{}}\PY{l+s}{Division is }\PY{l+s+si}{\PYZdl{}}\PY{p}{(}\PY{n}{calculator}\PY{p}{(}\PY{l+m+mi}{5}\PY{p}{,}\PY{+w}{ }\PY{l+m+mi}{3}\PY{p}{,}\PY{+w}{ }\PY{l+s+sc}{\PYZsq{}÷\PYZsq{}}\PY{p}{)}\PY{p}{)}\PY{l+s}{\PYZdq{}}\PY{p}{)}
\PY{n}{println}\PY{p}{(}\PY{n}{calculator}\PY{p}{(}\PY{l+m+mi}{5}\PY{p}{,}\PY{+w}{ }\PY{l+m+mi}{3}\PY{p}{,}\PY{+w}{ }\PY{l+s+sc}{\PYZsq{}×\PYZsq{}}\PY{p}{)}\PY{p}{)}
\end{Verbatim}
\end{tcolorbox}

    \begin{Verbatim}[commandchars=\\\{\}]
Summation is 8
Subtraction is 2
Multiplication is 15
Division is 1.6666666666666667
INVALID OPERATION
    \end{Verbatim}

    \textbf{Miscallenous}

    \begin{tcolorbox}[breakable, size=fbox, boxrule=1pt, pad at break*=1mm,colback=cellbackground, colframe=cellborder]
\prompt{In}{incolor}{122}{\boxspacing}
\begin{Verbatim}[commandchars=\\\{\}]
\PY{l+s+sa}{md}\PY{l+s}{\PYZdq{}}\PY{l+s}{Check your version of Julia}\PY{l+s}{\PYZdq{}}
\PY{n}{versioninfo}\PY{p}{(}\PY{p}{)}
\end{Verbatim}
\end{tcolorbox}

    \begin{Verbatim}[commandchars=\\\{\}]
Julia Version 1.9.3
Commit bed2cd540a (2023-08-24 14:43 UTC)
Build Info:

    Note: This is an unofficial build, please report bugs to the project
    responsible for this build and not to the Julia project unless you can
    reproduce the issue using official builds available at
https://julialang.org/downloads

Platform Info:
  OS: Linux (x86\_64-linux-gnu)
  CPU: 8 × Intel(R) Core(TM) i7-8565U CPU @ 1.80GHz
  WORD\_SIZE: 64
  LIBM: libopenlibm
  LLVM: libLLVM-14.0.6 (ORCJIT, skylake)
  Threads: 2 on 8 virtual cores
    \end{Verbatim}

    \begin{tcolorbox}[breakable, size=fbox, boxrule=1pt, pad at break*=1mm,colback=cellbackground, colframe=cellborder]
\prompt{In}{incolor}{123}{\boxspacing}
\begin{Verbatim}[commandchars=\\\{\}]
\PY{l+s+sa}{md}\PY{l+s}{\PYZdq{}}\PY{l+s}{The macro `@edit` shows the defintion of a function when invoked with specific arguments}\PY{l+s}{\PYZdq{}}
\PY{c}{\PYZsh{} @edit maximum([\PYZhy{}1, 0, 1])}
\end{Verbatim}
\end{tcolorbox}
 
            
\prompt{Out}{outcolor}{123}{}
    
    The macro \texttt{@edit} shows the defintion of a function when invoked with specific arguments



    

    \begin{tcolorbox}[breakable, size=fbox, boxrule=1pt, pad at break*=1mm,colback=cellbackground, colframe=cellborder]
\prompt{In}{incolor}{124}{\boxspacing}
\begin{Verbatim}[commandchars=\\\{\}]
\PY{l+s+sa}{md}\PY{l+s}{\PYZdq{}}\PY{l+s}{`varinfo` lists all global variables and their corresponding types in the current scope}\PY{l+s}{\PYZdq{}}
\PY{n}{varinfo}\PY{p}{(}\PY{p}{)}
\end{Verbatim}
\end{tcolorbox}
 
            
\prompt{Out}{outcolor}{124}{}
    
    \begin{tabular}
{l | r | l}
name & size & summary \\
\hline
Base &  & Module \\
Core &  & Module \\
Foo & 0 bytes & Foo (generic function with 1 method) \\
M & 200 bytes & 5×4 Matrix\{Float64\} \\
Main &  & Module \\
PLOTS\_DEFAULTS & 456 bytes & Dict\{Symbol, Symbol\} with 1 entry \\
a & 0 bytes & Irrational\{:π\} \\
b & 0 bytes & Irrational\{:π\} \\
bar & 0 bytes & bar (generic function with 1 method) \\
calculator & 0 bytes & calculator (generic function with 1 method) \\
col\_idx & 44 bytes & 4-element Vector\{Bool\} \\
df & 915 bytes & 4×3 DataFrame \\
dict & 503 bytes & Dict\{String, Int64\} with 3 entries \\
f & 0 bytes & f (generic function with 4 methods) \\
file & 372 bytes & IOStream \\
foo & 0 bytes & foo (generic function with 4 methods) \\
fruits & 160 bytes & 5-element Vector\{String\} \\
g & 0 bytes & g (generic function with 1 method) \\
h & 0 bytes & h (generic function with 1 method) \\
i & 8 bytes & Int64 \\
idx & 45 bytes & 5-element Vector\{Bool\} \\
iter & 8 bytes & Int64 \\
lst & 176 bytes & 5-element Vector\{Any\} \\
mycos & 0 bytes & mycos (generic function with 2 methods) \\
row\_idx & 45 bytes & 5-element Vector\{Bool\} \\
showall & 0 bytes & showall (generic function with 1 method) \\
str & 19 bytes & 11-codeunit String \\
tmp & 1 byte & UInt8 \\
var & 80 bytes & 2×4 BitMatrix \\
vegetables & 159 bytes & 5-element Vector\{String\} \\
x & 120 bytes & 5-element Vector\{ComplexF64\} \\
y & 781.289 KiB & 100000-element Vector\{Float64\} \\
z & 768 bytes & 91-element Vector\{Float64\} \\
\end{tabular}


    

    \emph{Modules}

    \begin{tcolorbox}[breakable, size=fbox, boxrule=1pt, pad at break*=1mm,colback=cellbackground, colframe=cellborder]
\prompt{In}{incolor}{125}{\boxspacing}
\begin{Verbatim}[commandchars=\\\{\}]
\PY{k}{module}\PY{+w}{ }\PY{n}{MyModule}
\PY{k}{export}\PY{+w}{ }\PY{n}{a}
\PY{n}{a}\PY{+w}{ }\PY{o}{=}\PY{+w}{ }\PY{l+m+mi}{0}
\PY{n}{b}\PY{+w}{ }\PY{o}{=}\PY{+w}{ }\PY{n+nb}{true}
\PY{k}{end}
\end{Verbatim}
\end{tcolorbox}

            \begin{tcolorbox}[breakable, size=fbox, boxrule=.5pt, pad at break*=1mm, opacityfill=0]
\prompt{Out}{outcolor}{125}{\boxspacing}
\begin{Verbatim}[commandchars=\\\{\}]
Main.MyModule
\end{Verbatim}
\end{tcolorbox}
        
    \begin{tcolorbox}[breakable, size=fbox, boxrule=1pt, pad at break*=1mm,colback=cellbackground, colframe=cellborder]
\prompt{In}{incolor}{126}{\boxspacing}
\begin{Verbatim}[commandchars=\\\{\}]
\PY{n}{varinfo}\PY{p}{(}\PY{n}{MyModule}\PY{p}{)}
\end{Verbatim}
\end{tcolorbox}
 
            
\prompt{Out}{outcolor}{126}{}
    
    \begin{tabular}
{l | r | l}
name & size & summary \\
\hline
MyModule & 2.304 KiB & Module \\
a & 8 bytes & Int64 \\
\end{tabular}


    

    \begin{tcolorbox}[breakable, size=fbox, boxrule=1pt, pad at break*=1mm,colback=cellbackground, colframe=cellborder]
\prompt{In}{incolor}{127}{\boxspacing}
\begin{Verbatim}[commandchars=\\\{\}]
\PY{n}{a}
\end{Verbatim}
\end{tcolorbox}

            \begin{tcolorbox}[breakable, size=fbox, boxrule=.5pt, pad at break*=1mm, opacityfill=0]
\prompt{Out}{outcolor}{127}{\boxspacing}
\begin{Verbatim}[commandchars=\\\{\}]
π = 3.1415926535897{\ldots}
\end{Verbatim}
\end{tcolorbox}
        
    \begin{tcolorbox}[breakable, size=fbox, boxrule=1pt, pad at break*=1mm,colback=cellbackground, colframe=cellborder]
\prompt{In}{incolor}{128}{\boxspacing}
\begin{Verbatim}[commandchars=\\\{\}]
\PY{n}{MyModule}\PY{o}{.}\PY{n}{a}
\end{Verbatim}
\end{tcolorbox}

            \begin{tcolorbox}[breakable, size=fbox, boxrule=.5pt, pad at break*=1mm, opacityfill=0]
\prompt{Out}{outcolor}{128}{\boxspacing}
\begin{Verbatim}[commandchars=\\\{\}]
0
\end{Verbatim}
\end{tcolorbox}
        
    \begin{tcolorbox}[breakable, size=fbox, boxrule=1pt, pad at break*=1mm,colback=cellbackground, colframe=cellborder]
\prompt{In}{incolor}{129}{\boxspacing}
\begin{Verbatim}[commandchars=\\\{\}]
\PY{n}{MyModule}\PY{o}{.}\PY{n}{b}
\end{Verbatim}
\end{tcolorbox}

            \begin{tcolorbox}[breakable, size=fbox, boxrule=.5pt, pad at break*=1mm, opacityfill=0]
\prompt{Out}{outcolor}{129}{\boxspacing}
\begin{Verbatim}[commandchars=\\\{\}]
true
\end{Verbatim}
\end{tcolorbox}
        
    \begin{tcolorbox}[breakable, size=fbox, boxrule=1pt, pad at break*=1mm,colback=cellbackground, colframe=cellborder]
\prompt{In}{incolor}{ }{\boxspacing}
\begin{Verbatim}[commandchars=\\\{\}]

\end{Verbatim}
\end{tcolorbox}


    % Add a bibliography block to the postdoc
    
    
    
\end{document}
